%%=============================================================================
%% Samenvatting
%%=============================================================================

% TODO: De "abstract" of samenvatting is een kernachtige (~ 1 blz. voor een
% thesis) synthese van het document.
%
% Een goede abstract biedt een kernachtig antwoord op volgende vragen:
%
% 1. Waarover gaat de bachelorproef?
% 2. Waarom heb je er over geschreven?
% 3. Hoe heb je het onderzoek uitgevoerd?
% 4. Wat waren de resultaten? Wat blijkt uit je onderzoek?
% 5. Wat betekenen je resultaten? Wat is de relevantie voor het werkveld?
%
% Daarom bestaat een abstract uit volgende componenten:
%
% - inleiding + kaderen thema
% - probleemstelling
% - (centrale) onderzoeksvraag
% - onderzoeksdoelstelling
% - methodologie
% - resultaten (beperk tot de belangrijkste, relevant voor de onderzoeksvraag)
% - conclusies, aanbevelingen, beperkingen
%
% LET OP! Een samenvatting is GEEN voorwoord!

%%---------- Nederlandse samenvatting -----------------------------------------
%
% TODO: Als je je bachelorproef in het Engels schrijft, moet je eerst een
% Nederlandse samenvatting invoegen. Haal daarvoor onderstaande code uit
% commentaar.
% Wie zijn bachelorproef in het Nederlands schrijft, kan dit negeren, de inhoud
% wordt niet in het document ingevoegd.

% \IfLanguageName{english}{
% \selectlanguage{dutch}
% \chapter*{Samenvatting}


% \selectlanguage{english}
% }{}
%%---------- Samenvatting -----------------------------------------------------
% De samenvatting in de hoofdtaal van het document
\chapter*{\IfLanguageName{dutch}{Samenvatting}{Abstract}}
In de fabriek van ArcelorMittal Gent worden dagelijks staalrollen geproduceerd die met een uniek nummer worden getraceerd. 
Toch is het vaak onduidelijk waar in het proces een fout is ontstaan of welke machines hier invloed op hebben. 
Dit leidt tot inefficiënte zoektochten, wat tijd en middelen kost.

Dit onderzoek zoekt een methode om procesfouten en andere use cases snel en accuraat te identificeren, zodat medewerkers snel informatie kunnen vinden over producten of gerelateerde middelen zoals kranen of machines. 
De bachelorproef is opgesteld in opdracht van Tracked, gespecialiseerd in het traceren van goederen en processen. 
ArcelorMittal Gent vroeg om een oplossing die procestracering versnelt en automatiseert, met als doel het zoeken naar informatie eenvoudiger te maken.

De bachelorproef bestaat uit drie delen. 
Het eerste deel is een literatuurstudie over technologieën als Cosmos DB en de Gremlin API, die het opzetten van een grafiekmodel mogelijk maken. 

Het tweede deel behandelt het vertalen van SAP-data naar een grafiekmodel in een grafiekdatabase. 
Hiervoor wordt een JSON-LD-bestand gebruikt, opgebouwd met schema.org en GS1 EPCIS-events. 
Dit bestand wordt via NodeJS en de Gremlin API ingeladen in Cosmos DB, wat resulteert in een grafiekmodel waarin data en relaties gevisualiseerd en geanalyseerd kunnen worden.

Het derde deel beschrijft de ontwikkeling van een chatbot die dit grafiekmodel kan bevragen en analyseren. 
De chatbot vertaalt gebruikersvragen naar Gremlin-queries die data uit Cosmos DB ophalen. 
De relevante knopen en relaties worden omgezet in leesbare tekst, waardoor de chatbot antwoorden kan geven op vragen zoals: ``Geef alle kranen met een melding op de motor.''
Dit hoofdstuk gaat ook in op het gebruik van Docker voor het beheren van de chatbot, in combinatie met Elasticsearch en Ollama-modellen.

De focus ligt op een beperkte set vragen, omdat een uitgebreide vragenset niet het hoofddoel is. 
Wel wordt een basis gelegd voor een chatbot die vragen over het productieproces van ArcelorMittal Gent kan beantwoorden, vooral gericht op producttracering en het opsporen van foutmeldingen.