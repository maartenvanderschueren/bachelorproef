%%=============================================================================
%% Samenvatting
%%=============================================================================

% TODO: De "abstract" of samenvatting is een kernachtige (~ 1 blz. voor een
% thesis) synthese van het document.
%
% Een goede abstract biedt een kernachtig antwoord op volgende vragen:
%
% 1. Waarover gaat de bachelorproef?
% 2. Waarom heb je er over geschreven?
% 3. Hoe heb je het onderzoek uitgevoerd?
% 4. Wat waren de resultaten? Wat blijkt uit je onderzoek?
% 5. Wat betekenen je resultaten? Wat is de relevantie voor het werkveld?
%
% Daarom bestaat een abstract uit volgende componenten:
%
% - inleiding + kaderen thema
% - probleemstelling
% - (centrale) onderzoeksvraag
% - onderzoeksdoelstelling
% - methodologie
% - resultaten (beperk tot de belangrijkste, relevant voor de onderzoeksvraag)
% - conclusies, aanbevelingen, beperkingen
%
% LET OP! Een samenvatting is GEEN voorwoord!

%%---------- Nederlandse samenvatting -----------------------------------------
%
% TODO: Als je je bachelorproef in het Engels schrijft, moet je eerst een
% Nederlandse samenvatting invoegen. Haal daarvoor onderstaande code uit
% commentaar.
% Wie zijn bachelorproef in het Nederlands schrijft, kan dit negeren, de inhoud
% wordt niet in het document ingevoegd.

% \IfLanguageName{english}{
% \selectlanguage{dutch}
% \chapter*{Samenvatting}


% \selectlanguage{english}
% }{}
%%---------- Samenvatting -----------------------------------------------------
% De samenvatting in de hoofdtaal van het document
\chapter*{\IfLanguageName{dutch}{Samenvatting}{Abstract}}
In de fabrieksomgeving van ArcelorMittal Gent worden er dagelijks verschillende staalplaten geproduceerd.
Deze platen worden doorheen het productieproces getraceerd door middel van een uniek nummer.
Het is echter niet altijd duidelijk waar in het productieproces een fout is ontstaan.
In deze bachelorproef wordt er onderzocht hoe we met behulp van een chatbot en grafiekmodellering efficiënt kunnen achterhalen waar in het staalproces een mogelijke fout is ontstaan.
Het doel van dit onderzoek is om een methode te vinden die het mogelijk maakt snel en accuraat procesfouten of andere cases te identificeren en te analyseren, waardoor een werknemer snel een antwoord kan vinden over een bepaald product.
Deze bachelorproef is opgesteld op aanvraag van Tracked, een bedrijf dat gespecialiseerd is in het traceren van goederen en processen.
ArcelorMittal Gent vroeg hen om een oplossing te vinden voor het versnellen en automatiseren van procestracering. 

Deze bachelorproef is opgedeeld in verschillende delen.

In het eerste deel wordt er een literatuurstudie uitgevoerd over de verschillende technologieën die gebruikt zullen worden in dit onderzoek.
Hier hebben we onder andere CosmosDB en Gremlin API ontdekt die ons zullen helpen bij het efficiënt opzetten van een grafiekmodel.
In het tweede deel gaan we dieper in op de data en het vertalen van SAP data naar een grafiekmodel.
Dit gebeurt door de JSON data die we ontvangen hebben van ArcelorMittal Gent om te zetten naar een JSON-LD die GS1 standaarden bevat.
Daarna wordt deze data ingeladen in CosmosDB via NodeJS en Gremlin API.\@
In het derde deel wordt er een chatbot opgesteld die de data van het grafiekmodel kan bevragen en analyseren. 
Daardoor kan de chatbot antwoorden geven op vragen over het productieproces.

In deze bachelorproef gaan we ons focussen op een klein aantal vragen die gesteld kunnen worden omdat dit niet scope is van het onderzoek en de gekregen data beperkt is.
Deze vragen zullen voornamelijk gaan over het traceren van een product en het opzoeken van foutmeldingen in het productieproces.