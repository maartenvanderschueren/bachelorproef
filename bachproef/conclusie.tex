%%=============================================================================
%% Conclusie
%%=============================================================================

\chapter{Conclusie}%
\label{ch:conclusie}
In deze bachelorproef werd onderzocht hoe we gebruik konden maken van een grafiekmodel en een Large Language Model (LLM) om gegevens binnen het staalproductieproces van ArcelorMittal Gent efficiënt op te vragen. 
Het doel was om een proof of concept te ontwikkelen die niet alleen de traceerbaarheid van productieprocessen verbetert, maar ook de operationele efficiëntie verhoogt door middel van een chatbot die procesvragen kan beantwoorden.
Door de SAP boomstructuur om te zetten naar een JSON-LD formaat volgens GS1-standaarden, en deze te structureren in een grafiekdatabase (CosmosDB), werd een stevige basis gelegd voor het traceren van productieprocessen.
Vervolgens werd een chatbot ontwikkeld die gebruik maakt van Gremlin queries en Retrieval Augmented Generation (RAG) om procesvragen te vertalen naar begrijpelijke antwoorden.
Door gebruik te maken van modellen zoals Code Llama voor querygeneratie en Phi-4 voor natuurlijke taalverwerking, kon een performante oplossing worden opgezet die uit te bereiden is en bruikbaar voor andere use-cases binnen de industrie.

In dit ondezoek ben ik vooral op zoek gegaan naar combinaties van technieken die het mogelijk maken om de traceerbaarheid van productieprocessen te verbeteren.
De proof of concept die werd ontwikkeld, toont aan dat het mogelijk is om een chatbot te creëren die vragen kan beantwoorden over productieprocessen door gebruik te maken van dit grafiekmodel en een LLM.
Mits de juiste finetuning en eventuele hertraining van de modellen, kan deze oplossing verder worden uitgebreid en geoptimaliseerd voor andere toepassingen binnen de industrie.
Naast de technische aspecten, is het ook belangrijk om de gebruikerservaring in overweging te nemen.
De chatbot is een API die kan worden geïntegreerd in bestaande systemen, waardoor het voor gebruikers eenvoudig is om toegang te krijgen tot de informatie die ze nodig hebben.
De proof of concept biedt een solide basis voor verdere ontwikkeling en implementatie binnen ArcelorMittal Gent en mogelijk ook andere bedrijven in de staalindustrie.
Naast het opzetten van een frontend, kan er ook gekeken worden naar het maken van een gremlin specifieke LLM die getrained wordt op de mogelijke syntax van gremlin queries.
In deze thesis hebben we daar niet de tijd en resources voor gehad, maar het lijkt ons een interessante richting om verder te onderzoeken.

% TODO: Trek een duidelijke conclusie, in de vorm van een antwoord op de
% onderzoeksvra(a)g(en). Wat was jouw bijdrage aan het onderzoeksdomein en
% hoe biedt dit meerwaarde aan het vakgebied/doelgroep? 
% Reflecteer kritisch over het resultaat. In Engelse teksten wordt deze sectie
% ``Discussion'' genoemd. Had je deze uitkomst verwacht? Zijn er zaken die nog
% niet duidelijk zijn?
% Heeft het onderzoek geleid tot nieuwe vragen die uitnodigen tot verder 
%onderzoek?


