%%=============================================================================
%% Voorwoord
%%=============================================================================

\chapter*{\IfLanguageName{dutch}{Woord vooraf}{Preface}}%
\label{ch:voorwoord}

%% TODO:
%% Het voorwoord is het enige deel van de bachelorproef waar je vanuit je
%% eigen standpunt (``ik-vorm'') mag schrijven. Je kan hier bv. motiveren
%% waarom jij het onderwerp wil bespreken.
%% Vergeet ook niet te bedanken wie je geholpen/gesteund/... heeft
In deze bachelorproef heb ik me verdiept in het ontwikkelen van een oplossing waarmee we met behulp van een chatbot en grafiekmodellering efficiënt kunnen achterhalen waar in het staalproces van ArcelorMittal een mogelijke fout is ontstaan. 
Dit staalproces is een complex stappenplan waarbij het staal begint bij grondstoffen die in de fabriek aankomen en eindigen als een afgewerkt product dat naar de klant wordt verzonden.
Het doel van dit onderzoek is om een methode te vinden die het mogelijk maakt snel en accuraat processen te identificeren en eventuele foutmeldingen te detecteren.
Door een vraag te stellen aan de chatbot, gaat deze op zoek naar mogelijke knelpunten in het proces. Daarna verwachten we dat de chatbot dit in een duidelijk antwoord kan uitleggen op basis van de gegevens uit de database.
Dit onderzoek voer ik uit in samenwerking met mijn co-promotor Bart Peirens van Tracked, die ons ondersteunt bij het traceerproces. Daarnaast zal Tim De Grave, software architect binnen ArcelorMittal Gent de benodigde data en resources aanleveren waarmee we deze oplossing kunnen verwezelijken. 
Ik wil hen beiden hartelijk bedanken voor hun waardevolle bijdragen en ondersteuning gedurende dit project. Verder wil ik Martijn Saelens, mijn promotor, bedanken voor de begeleiding en het vertrouwen dat hij me gegeven heeft tijdens dit proces.
