%%=============================================================================
%% Voorwoord
%%=============================================================================

\chapter*{\IfLanguageName{dutch}{Woord vooraf}{Preface}}%
\label{ch:voorwoord}

%% TODO:
%% Het voorwoord is het enige deel van de bachelorproef waar je vanuit je
%% eigen standpunt (``ik-vorm'') mag schrijven. Je kan hier bv. motiveren
%% waarom jij het onderwerp wil bespreken.
%% Vergeet ook niet te bedanken wie je geholpen/gesteund/... heeft
In deze bachelorproef heb ik me verdiept in het ontwikkelen van een oplossing waarmee er met behulp van een chatbot en grafiekmodellering, efficiënt kan worden achterhaald waar in het staalverwerkingsproces van ArcelorMittal een mogelijke fout is ontstaan en of deze fouten kunnen worden voorkomen door foutmeldingen op te zoeken. 
Dit staalverwerkingsproces is een complex traject waarbij het staal start als grondstof die in de fabriek aankomt en eindigt als een afgewerkt product dat naar de klant wordt verzonden.
Het doel van dit onderzoek is om een methode te ontwikkelen die deze foutmeldingen kan weergeven op basis van een vraag gesteld door een gebruiker.
Door deze vraag te gebruiken in een chatbot, zal deze op zijn beurt op zoek gaan naar een antwoord binnen het proces. Daarna verwachten we dat de chatbot dit in een duidelijke samenvatting als antwoord kan geven op basis van de gegevens uit de database.
Dit onderzoek voer ik uit in samenwerking met mijn co-promotor Bart Peirens van Tracked, die ons ondersteuning biedt bij het traceerproces. 
Daarnaast zal Tim De Grave (softwarearchitect binnen ArcelorMittal Gent) de nodige data aanleveren, waarmee we deze oplossing kunnen realiseren. 
Ik wil hen beiden hartelijk bedanken voor hun waardevolle bijdragen en ondersteuning gedurende dit project. Verder wil ik ook mijn promotor, Martijn Saelens, bedanken voor de begeleiding en het vertrouwen dat hij me gegeven heeft tijdens dit traject.
