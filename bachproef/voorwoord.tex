%%=============================================================================
%% Voorwoord
%%=============================================================================

\chapter*{\IfLanguageName{dutch}{Woord vooraf}{Preface}}%
\label{ch:voorwoord}

%% TODO:
%% Het voorwoord is het enige deel van de bachelorproef waar je vanuit je
%% eigen standpunt (``ik-vorm'') mag schrijven. Je kan hier bv. motiveren
%% waarom jij het onderwerp wil bespreken.
%% Vergeet ook niet te bedanken wie je geholpen/gesteund/... heeft
In deze bachelorproef heb ik me verdiept in het ontwikkelen van een oplossing waarmee we met behulp van een chatbot en grafiekmodellering efficiënt kunnen achterhalen waar in het staalproces van ArcelorMittal een mogelijke fout is ontstaan. 
Het doel van dit onderzoek is om een methodiek te vinden die het mogelijk maakt snel en accuraat procesfouten te identificeren en te analyseren.
Door een vraag te stellen aan de chatbot, gaat deze op zoek naar mogelijke anomalieën in het proces. Daarna verwachten we dat de chatbot dit in een duidelijk antwoord kan uitleggen
Dit onderzoek voer ik uit in samenwerking met meneer Bart Peirens van Tracked N.V., die ons ondersteunt bij het traceren van de fouten. Daarnaast zal Tim De Grave van ArcelorMittal de benodigde data aanleveren waarop we deze oplossing kunnen toepassen. 
Ik wil hen beiden hartelijk bedanken voor hun waardevolle bijdragen en ondersteuning gedurende dit project. Verder wil ik Martijn Saelens, mijn promotor, bedanken voor de begeleiding en het vertrouwen dat hij me gegeven heeft tijdens dit proces.
