%%=============================================================================
%% Inleiding
%%=============================================================================

\chapter{\IfLanguageName{dutch}{Inleiding}{Introduction}}%
\label{ch:inleiding}

In de hedendaagse industriële wereld is efficiëntie van cruciaal belang, vooral binnen complexe toeleveringsketens zoals die van ArcelorMittal Gent. 
Deze bachelorproef richt zich op het verbeteren van klachtenafhandeling door gebruik te maken van geavanceerde technologieën zoals grafiekmodellering en large language models (LLM). 
Door traceringsdata om te zetten naar een grafiekmodel en dit te koppelen aan een LLM, willen we bedrijven in staat stellen sneller en nauwkeuriger de oorzaken van problemen te achterhalen. 
Dit onderzoek, uitgevoerd in samenwerking met Tracked, een expertisecentrum binnen de Cronos groep, heeft als doel om schaalbare en performante oplossingen te bieden die de operationele efficiëntie en klanttevredenheid optimaliseren.
Voor dit onderzoek gebruiken we een scope van één week data op een deel van het proces, waarna bij slagen dit proces kan uitgebreid worden naar andere afdelingen of bedrijven.

\section{\IfLanguageName{dutch}{Probleemstelling}{Problem Statement}}%
\label{sec:probleemstelling}

% Uit je probleemstelling moet duidelijk zijn dat je onderzoek een meerwaarde heeft voor een concrete doelgroep. De doelgroep moet goed gedefinieerd en afgelijnd zijn. Doelgroepen als ``bedrijven,'' ``KMO's'', systeembeheerders, enz.~zijn nog te vaag. Als je een lijstje kan maken van de personen/organisaties die een meerwaarde zullen vinden in deze bachelorproef (dit is eigenlijk je steekproefkader), dan is dat een indicatie dat de doelgroep goed gedefinieerd is. Dit kan een enkel bedrijf zijn of zelfs één persoon (je co-promotor/opdrachtgever).
In de complexe toeleveringsketen van ArcelorMittal Gent zijn er veel verschillende processen die verspreid zijn over meerdere afdelingen. 
Elke afdeling heeft specifieke processen en datamodellen, wat het analyseren en onderzoeken van gegevens over afdelingen heen zeer complex en tijdrovend maakt. 
Dit leidt tot inefficiënties in de klachtenafhandeling en verhoogt de operationele kosten. 
Er is behoefte aan een schaalbare en performante oplossing die bedrijven in staat stelt sneller en nauwkeuriger de oorzaken van problemen te achterhalen, waardoor kostbaar handmatig werk wordt verminderd.
Dit is een erkend probleem voor zowel kleine bedrijven bijvoorbeeld KMO's of delen van grote bedrijven zoals in ons geval ArcelorMittal Gent. Ook voor grote internationale bedrijven zoals ArcelorMittal international waarbij het staal een grote weg aflegt en niet alleen in Gent blijft.

\section{\IfLanguageName{dutch}{Onderzoeksvraag}{Research question}}%
\label{sec:onderzoeksvraag}

% Wees zo concreet mogelijk bij het formuleren van je onderzoeksvraag. Een onderzoeksvraag is trouwens iets waar nog niemand op dit moment een antwoord heeft (voor zover je kan nagaan). Het opzoeken van bestaande informatie (bv. ``welke tools bestaan er voor deze toepassing?'') is dus geen onderzoeksvraag. Je kan de onderzoeksvraag verder specifiëren in deelvragen. Bv.~als je onderzoek gaat over performantiemetingen, dan 
Hoe kunnen we efficiënt en snel grafiekmodellering toepassen om een large language model (LLM) te ontwikkelen dat in staat is om het waar, wanneer, wat en hoe van gebeurtenissen binnen een proces vast te stellen, ter ondersteuning van klachtenafhandeling bij productfouten?

\section{\IfLanguageName{dutch}{Onderzoeksdoelstelling}{Research objective}}%
\label{sec:onderzoeksdoelstelling}

% Wat is het beoogde resultaat van je bachelorproef? Wat zijn de criteria voor succes? Beschrijf die zo concreet mogelijk. Gaat het bv.\ om een proof-of-concept, een prototype, een verslag met aanbevelingen, een vergelijkende studie, enz.
De doelstelling van deze bachelorproef is om een schaalbare en performante oplossing te ontwikkelen die bedrijven in staat stelt sneller en nauwkeuriger de oorzaken van problemen binnen hun toeleveringsketen te achterhalen. 
Dit wordt bereikt door het toepassen van grafiekmodellering op traceringsdata en het trainen van een large language model (LLM). 
Het uiteindelijke doel is om de operationele efficiëntie te verbeteren en de klanttevredenheid te optimaliseren door een efficiënte klachtenafhandeling mogelijk te maken.

\section{\IfLanguageName{dutch}{Opzet van deze bachelorproef}{Structure of this bachelor thesis}}%
\label{sec:opzet-bachelorproef}

% Het is gebruikelijk aan het einde van de inleiding een overzicht te
% geven van de opbouw van de rest van de tekst. Deze sectie bevat al een aanzet
% die je kan aanvullen/aanpassen in functie van je eigen tekst.

De rest van deze bachelorproef is als volgt opgebouwd:

In Hoofdstuk~\ref{ch:stand-van-zaken} wordt een overzicht gegeven van de stand van zaken binnen het onderzoeksdomein, op basis van een literatuurstudie.

In Hoofdstuk~\ref{ch:methodologie} wordt de methodologie toegelicht en worden de gebruikte onderzoekstechnieken besproken om een antwoord te kunnen formuleren op de onderzoeksvragen.

% TODO: Vul hier aan voor je eigen hoofstukken, één of twee zinnen per hoofdstuk

In Hoofdstuk~\ref{ch:conclusie}, tenslotte, wordt de conclusie gegeven en een antwoord geformuleerd op de onderzoeksvragen. Daarbij wordt ook een aanzet gegeven voor toekomstig onderzoek binnen dit domein.