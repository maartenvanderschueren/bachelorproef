%===============================================================================
% LaTeX sjabloon voor de bachelorproef toegepaste informatica aan HOGENT
% Meer info op https://github.com/HoGentTIN/latex-hogent-report
%===============================================================================

\documentclass[dutch,dit,thesis]{hogentreport}

% TODO:
% - If necessary, replace the option `dit`' with your own department!
%   Valid entries are dbo, dbt, dgz, dit, dlo, dog, dsa, soa
% - If you write your thesis in English (remark: only possible after getting
%   explicit approval!), remove the option "dutch," or replace with "english".

%% Pictures to include in the text can be put in the graphics/ folder
\graphicspath{{../graphics/}}

%% For source code highlighting, requires pygments to be installed
%% Compile with the -shell-escape flag!
\usepackage[chapter]{minted}
%% If you compile with the make_thesis.{bat,sh} script, use the following
%% import instead:
% \usepackage[chapter,outputdir=../output]{minted}
% \usemintedstyle{solarized-light}

%% Formatting for minted environments.
\setminted{%
    autogobble,
    frame=lines,
    breaklines,
    linenos,
    tabsize=4
}

%% Ensure the list of listings is in the table of contents
\renewcommand\listoflistingscaption{%
    \IfLanguageName{dutch}{Lijst van codefragmenten}{List of listings}
}
\renewcommand\listingscaption{%
    \IfLanguageName{dutch}{Codefragment}{Listing}
}
\renewcommand*\listoflistings{%
    \cleardoublepage\phantomsection\addcontentsline{toc}{chapter}{\listoflistingscaption}%
    \listof{listing}{\listoflistingscaption}%
}

% Other packages not already included can be imported here

%%---------- Document metadata -------------------------------------------------
% TODO: Replace this with your own information
\author{Maarten Van der Schueren}
\supervisor{Dhr. M. Saelens}
\cosupervisor{Dhr. B. Peirens}
\title[]%
    {Toepassen van een tijdelijke \newline grafiektransformatie op object traceringsdata voor het trainen van een LLM}
\academicyear{\advance\year by -1 \the\year--\advance\year by 1 \the\year}
\examperiod{1}
\degreesought{\IfLanguageName{dutch}{Professionele bachelor in de toegepaste informatica}{Bachelor of applied computer science}}
\partialthesis{false} %% To display 'in partial fulfilment'
\institution{Tracked}

%% Add global exceptions to the hyphenation here
\hyphenation{back-slash}

%% The bibliography (style and settings are  found in hogentthesis.cls)
\addbibresource{bachproef.bib}           %% Bibliography file
\addbibresource{../voorstel/voorstel.bib} %% Bibliography research proposal
\defbibheading{bibempty}{}

%% Prevent empty pages for right-handed chapter starts in twoside mode
\renewcommand{\cleardoublepage}{\clearpage}

\renewcommand{\arraystretch}{1.2}

%% Content starts here.
\begin{document}

%---------- Front matter -------------------------------------------------------

\frontmatter

\hypersetup{pageanchor=false} %% Disable page numbering references
%% Render a Dutch outer title page if the main language is English
\IfLanguageName{english}{%
    %% If necessary, information can be changed here
    \degreesought{Professionele Bachelor toegepaste informatica}%
    \begin{otherlanguage}{dutch}%
       \maketitle%
    \end{otherlanguage}%
}{}

%% Generates title page content
\maketitle
\hypersetup{pageanchor=true}

%%=============================================================================
%% Voorwoord
%%=============================================================================

\chapter*{\IfLanguageName{dutch}{Woord vooraf}{Preface}}%
\label{ch:voorwoord}

%% TODO:
%% Het voorwoord is het enige deel van de bachelorproef waar je vanuit je
%% eigen standpunt (``ik-vorm'') mag schrijven. Je kan hier bv. motiveren
%% waarom jij het onderwerp wil bespreken.
%% Vergeet ook niet te bedanken wie je geholpen/gesteund/... heeft
In deze bachelorproef heb ik me verdiept in het ontwikkelen van een oplossing waarmee we met behulp van een chatbot en grafiekmodellering efficiënt kunnen achterhalen waar in het staalproces van ArcelorMittal een mogelijke fout is ontstaan. 
Dit staalproces is een complex stappenplan waarbij het staal begint bij grondstoffen die in de fabriek aankomen en eindigen als een afgewerkt product dat naar de klant wordt verzonden.
Het doel van dit onderzoek is om een methode te vinden die het mogelijk maakt snel en accuraat processen te identificeren en eventuele foutmeldingen te detecteren.
Door een vraag te stellen aan de chatbot, gaat deze op zoek naar mogelijke knelpunten in het proces. Daarna verwachten we dat de chatbot dit in een duidelijk antwoord kan uitleggen op basis van de gegevens uit de database.
Dit onderzoek voer ik uit in samenwerking met mijn co-promotor Bart Peirens van Tracked, die ons ondersteunt bij het traceerproces. Daarnaast zal Tim De Grave, software architect binnen ArcelorMittal Gent de benodigde data en resources aanleveren waarmee we deze oplossing kunnen verwezelijken. 
Ik wil hen beiden hartelijk bedanken voor hun waardevolle bijdragen en ondersteuning gedurende dit project. Verder wil ik Martijn Saelens, mijn promotor, bedanken voor de begeleiding en het vertrouwen dat hij me gegeven heeft tijdens dit proces.

%%=============================================================================
%% Samenvatting
%%=============================================================================

% TODO: De "abstract" of samenvatting is een kernachtige (~ 1 blz. voor een
% thesis) synthese van het document.
%
% Een goede abstract biedt een kernachtig antwoord op volgende vragen:
%
% 1. Waarover gaat de bachelorproef?
% 2. Waarom heb je er over geschreven?
% 3. Hoe heb je het onderzoek uitgevoerd?
% 4. Wat waren de resultaten? Wat blijkt uit je onderzoek?
% 5. Wat betekenen je resultaten? Wat is de relevantie voor het werkveld?
%
% Daarom bestaat een abstract uit volgende componenten:
%
% - inleiding + kaderen thema
% - probleemstelling
% - (centrale) onderzoeksvraag
% - onderzoeksdoelstelling
% - methodologie
% - resultaten (beperk tot de belangrijkste, relevant voor de onderzoeksvraag)
% - conclusies, aanbevelingen, beperkingen
%
% LET OP! Een samenvatting is GEEN voorwoord!

%%---------- Nederlandse samenvatting -----------------------------------------
%
% TODO: Als je je bachelorproef in het Engels schrijft, moet je eerst een
% Nederlandse samenvatting invoegen. Haal daarvoor onderstaande code uit
% commentaar.
% Wie zijn bachelorproef in het Nederlands schrijft, kan dit negeren, de inhoud
% wordt niet in het document ingevoegd.

% \IfLanguageName{english}{
% \selectlanguage{dutch}
% \chapter*{Samenvatting}


% \selectlanguage{english}
% }{}
%%---------- Samenvatting -----------------------------------------------------
% De samenvatting in de hoofdtaal van het document
\chapter*{\IfLanguageName{dutch}{Samenvatting}{Abstract}}
In de fabrieksomgeving van ArcelorMittal Gent worden er dagelijks verschillende staalplaten geproduceerd.
Deze platen worden doorheen het productieproces getraceerd door middel van een uniek nummer.
Het is echter niet altijd duidelijk waar in het productieproces een fout is ontstaan.
In deze bachelorproef wordt er onderzocht hoe we met behulp van een chatbot en grafiekmodellering efficiënt kunnen achterhalen waar in het staalproces een mogelijke fout is ontstaan.
Het doel van dit onderzoek is om een methode te vinden die het mogelijk maakt snel en accuraat procesfouten of andere cases te identificeren en te analyseren, waardoor een werknemer snel een antwoord kan vinden over een bepaald product.
Deze bachelorproef is opgesteld op aanvraag van Tracked N.V., een bedrijf dat gespecialiseerd is in het traceren van goederen en processen.
ArcelorMittal Gent vroeg hen om een oplossing te vinden voor het versnellen en automatiseren van procestracering. 

Deze bachelorproef is opgedeeld in verschillende delen.
In het eerste deel wordt er een literatuurstudie uitgevoerd over de verschillende technologieën die gebruikt zullen worden in dit onderzoek.
Hier hebben we onder andere CosmosDB en Gremlin API ontdekt die ons zullen helpen bij het efficiënt opzetten van een grafiekmodel.
In het tweede deel wordt er een proof of concept opgesteld van het grafiekmodel waarbij we de data van ArcelorMittal Gent zullen gebruiken.
Dit gebeurt via een jsonLD-bestand dat we via NodeJS en Gremlin API kunnen inladen in CosmosDB.\@ Daarna kunnen we met visualisatie tools zoals Graph Explorer de data visualiseren.
In het derde deel wordt er een chatbot opgesteld die de data van het grafiekmodel kan bevragen en analyseren. 
Daardoor kan de chatbot antwoorden geven op vragen over het productieproces.
In deze bachelorproef gaan we ons focussen op een klein aantal vragen die gesteld kunnen worden omdat dit niet de hoofdzaak is van het onderzoek.
Deze vragen zullen voornamelijk gaan over het traceren van een product en het opzoeken van fouten in het productieproces.

%---------- Inhoud, lijst figuren, ... -----------------------------------------

\tableofcontents

% In a list of figures, the complete caption will be included. To prevent this,
% ALWAYS add a short description in the caption!
%
%  \caption[short description]{elaborate description}
%
% If you do, only the short description will be used in the list of figures

\listoffigures

% If you included tables and/or source code listings, uncomment the appropriate
% lines.
\listoftables

\listoflistings

% Als je een lijst van afkortingen of termen wil toevoegen, dan hoort die
% hier thuis. Gebruik bijvoorbeeld de ``glossaries'' package.
% https://www.overleaf.com/learn/latex/Glossaries

%---------- Kern ---------------------------------------------------------------

\mainmatter{}

% De eerste hoofdstukken van een bachelorproef zijn meestal een inleiding op
% het onderwerp, literatuurstudie en verantwoording methodologie.
% Aarzel niet om een meer beschrijvende titel aan deze hoofdstukken te geven of
% om bijvoorbeeld de inleiding en/of stand van zaken over meerdere hoofdstukken
% te verspreiden!

%%=============================================================================
%% Inleiding
%%=============================================================================

\chapter{\IfLanguageName{dutch}{Inleiding}{Introduction}}%
\label{ch:inleiding}

In de hedendaagse industriële wereld is efficiëntie van cruciaal belang, vooral binnen complexe toeleveringsketens zoals die van ArcelorMittal Gent. 
Deze bachelorproef richt zich op het verbeteren van klachtenafhandeling door gebruik te maken van geavanceerde technologieën zoals grafiekmodellering en large language models (LLM). 
Door traceringsdata om te zetten naar een grafiekmodel en dit te koppelen aan een LLM, willen we bedrijven in staat stellen sneller en nauwkeuriger de oorzaken van problemen te achterhalen. 
Dit onderzoek, uitgevoerd in samenwerking met Tracked, een expertisecentrum binnen de Cronos groep, heeft als doel om schaalbare en performante oplossingen te bieden die de operationele efficiëntie en klanttevredenheid optimaliseren.
Voor dit onderzoek gebruiken we een scope van één week data op een deel van het proces, waarna bij slagen dit proces kan uitgebreid worden naar andere afdelingen of bedrijven.

\section{\IfLanguageName{dutch}{Probleemstelling}{Problem Statement}}%
\label{sec:probleemstelling}

% Uit je probleemstelling moet duidelijk zijn dat je onderzoek een meerwaarde heeft voor een concrete doelgroep. De doelgroep moet goed gedefinieerd en afgelijnd zijn. Doelgroepen als ``bedrijven,'' ``KMO's'', systeembeheerders, enz.~zijn nog te vaag. Als je een lijstje kan maken van de personen/organisaties die een meerwaarde zullen vinden in deze bachelorproef (dit is eigenlijk je steekproefkader), dan is dat een indicatie dat de doelgroep goed gedefinieerd is. Dit kan een enkel bedrijf zijn of zelfs één persoon (je co-promotor/opdrachtgever).
In de complexe toeleveringsketen van ArcelorMittal Gent zijn er veel verschillende processen die verspreid zijn over meerdere afdelingen. 
Elke afdeling heeft specifieke processen en datamodellen, wat het analyseren en onderzoeken van gegevens over afdelingen heen zeer complex en tijdrovend maakt. 
Dit leidt tot inefficiënties in de klachtenafhandeling en verhoogt de operationele kosten. 
Er is behoefte aan een schaalbare en performante oplossing die bedrijven in staat stelt sneller en nauwkeuriger de oorzaken van problemen te achterhalen, waardoor kostbaar handmatig werk wordt verminderd.
Dit is een erkend probleem voor zowel kleine bedrijven bijvoorbeeld KMO's of delen van grote bedrijven zoals in ons geval ArcelorMittal Gent. Ook voor grote internationale bedrijven zoals ArcelorMittal international waarbij het staal een grote weg aflegt en niet alleen in Gent blijft.

\section{\IfLanguageName{dutch}{Onderzoeksvraag}{Research question}}%
\label{sec:onderzoeksvraag}

% Wees zo concreet mogelijk bij het formuleren van je onderzoeksvraag. Een onderzoeksvraag is trouwens iets waar nog niemand op dit moment een antwoord heeft (voor zover je kan nagaan). Het opzoeken van bestaande informatie (bv. ``welke tools bestaan er voor deze toepassing?'') is dus geen onderzoeksvraag. Je kan de onderzoeksvraag verder specifiëren in deelvragen. Bv.~als je onderzoek gaat over performantiemetingen, dan 
Hoe kunnen we efficiënt en snel grafiekmodellering toepassen om een large language model (LLM) te ontwikkelen dat in staat is om het waar, wanneer, wat en hoe van gebeurtenissen binnen een proces vast te stellen, ter ondersteuning van klachtenafhandeling bij productfouten?

\section{\IfLanguageName{dutch}{Onderzoeksdoelstelling}{Research objective}}%
\label{sec:onderzoeksdoelstelling}

% Wat is het beoogde resultaat van je bachelorproef? Wat zijn de criteria voor succes? Beschrijf die zo concreet mogelijk. Gaat het bv.\ om een proof-of-concept, een prototype, een verslag met aanbevelingen, een vergelijkende studie, enz.
De doelstelling van deze bachelorproef is om een schaalbare en performante oplossing te ontwikkelen die bedrijven in staat stelt sneller en nauwkeuriger de oorzaken van problemen binnen hun toeleveringsketen te achterhalen. 
Dit wordt bereikt door het toepassen van grafiekmodellering op traceringsdata en het trainen van een large language model (LLM). 
Het uiteindelijke doel is om de operationele efficiëntie te verbeteren en de klanttevredenheid te optimaliseren door een efficiënte klachtenafhandeling mogelijk te maken.

\section{\IfLanguageName{dutch}{Opzet van deze bachelorproef}{Structure of this bachelor thesis}}%
\label{sec:opzet-bachelorproef}

% Het is gebruikelijk aan het einde van de inleiding een overzicht te
% geven van de opbouw van de rest van de tekst. Deze sectie bevat al een aanzet
% die je kan aanvullen/aanpassen in functie van je eigen tekst.

De rest van deze bachelorproef is als volgt opgebouwd:

In Hoofdstuk~\ref{ch:stand-van-zaken} wordt een overzicht gegeven van de stand van zaken binnen het onderzoeksdomein, op basis van een literatuurstudie.

In Hoofdstuk~\ref{ch:methodologie} wordt de methodologie toegelicht en worden de gebruikte onderzoekstechnieken besproken om een antwoord te kunnen formuleren op de onderzoeksvragen.

% TODO: Vul hier aan voor je eigen hoofstukken, één of twee zinnen per hoofdstuk

In Hoofdstuk~\ref{ch:conclusie}, tenslotte, wordt de conclusie gegeven en een antwoord geformuleerd op de onderzoeksvragen. Daarbij wordt ook een aanzet gegeven voor toekomstig onderzoek binnen dit domein.
\chapter{\IfLanguageName{dutch}{Stand van zaken}{State of the art}}%
\label{ch:stand-van-zaken}

% Tip: Begin elk hoofdstuk met een paragraaf inleiding die beschrijft hoe
% dit hoofdstuk past binnen het geheel van de bachelorproef. Geef in het
% bijzonder aan wat de link is met het vorige en volgende hoofdstuk.

% Pas na deze inleidende paragraaf komt de eerste sectiehoofding.

% Dit hoofdstuk bevat je literatuurstudie. De inhoud gaat verder op de inleiding, maar zal het onderwerp van de bachelorproef *diepgaand* uitspitten. De bedoeling is dat de lezer na lezing van dit hoofdstuk helemaal op de hoogte is van de huidige stand van zaken (state-of-the-art) in het onderzoeksdomein. Iemand die niet vertrouwd is met het onderwerp, weet nu voldoende om de rest van het verhaal te kunnen volgen, zonder dat die er nog andere informatie moet over opzoeken \autocite{Pollefliet2011}.

% Je verwijst bij elke bewering die je doet, vakterm die je introduceert, enz.\ naar je bronnen. In \LaTeX{} kan dat met het commando \texttt{$\backslash${textcite\{\}}} of \texttt{$\backslash${autocite\{\}}}. Als argument van het commando geef je de ``sleutel'' van een ``record'' in een bibliografische databank in het Bib\LaTeX{}-formaat (een tekstbestand). Als je expliciet naar de auteur verwijst in de zin (narratieve referentie), gebruik je \texttt{$\backslash${}textcite\{\}}. Soms is de auteursnaam niet expliciet een onderdeel van de zin, dan gebruik je \texttt{$\backslash${}autocite\{\}} (referentie tussen haakjes). Dit gebruik je bv.~bij een citaat, of om in het bijschrift van een overgenomen afbeelding, broncode, tabel, enz. te verwijzen naar de bron. In de volgende paragraaf een voorbeeld van elk.

% \textcite{Knuth1998} schreef een van de standaardwerken over sorteer- en zoekalgoritmen. Experten zijn het erover eens dat cloud computing een interessante opportuniteit vormen, zowel voor gebruikers als voor dienstverleners op vlak van informatietechnologie~\autocite{Creeger2009}.

% Let er ook op: het \texttt{cite}-commando voor de punt, dus binnen de zin. Je verwijst meteen naar een bron in de eerste zin die erop gebaseerd is, dus niet pas op het einde van een paragraaf.

\label{sec:grafiekmodellering}
\section{Grafiekmodellering}
Grafiekmodellering is een techniek die gebruikt wordt om de data te visualiseren en te analyseren. In deze bachelorproef wordt dit gebruikt om de verbanden te leggen tussen de verschillende processen binnen ArcelorMittal Gent.
Dit gebeurt door middel van knopen die verbonden zijn met andere knopen met een relatie. \autocite{neo4j20252}.
Een knoop stelt een entiteit voor, zoals een persoon, een product of een proces. Een verbinding stelt de relatie tussen de verschillende knopen voor zoals een associatie, transformatie of transactie van een product.
Dit kan in ons geval een staalplaat zijn die door een kraan verplaatst wordt. Hierbij zijn de kraan en staalplaat de knopen en is de verplaatsing een transactie-relatie tussen deze knopen.
Elke knoop bevat ook properties die de knoop beschrijven. Dit kan bijvoorbeeld het bouwjaar van een machine zijn of de temperatuur van een product, deze worden opgeslagen als sleutel-waarde paar om later efficiënt te kunnen ophalen.

\subsection{Cosmos DB}%
Cosmos DB is een NoSQL-database van Microsoft. Het biedt een lage latentie, multi-query-API die eenvoudig grote hoeveelheden data kan verwerken en heeft een grote beschikbaarheid, zegt~\textcite{Put2020}, wat zeer belangrijk is in ons project.
Daarnaast is CosmosDB horizontaal schaalbaar, wat betekent dat we op hoogtepunten tot een miljoen lees- en schrijfaanvragen kunnen verwerken door het benodigde aantal servers toe te voegen.
De hoge beschikbaarheid wordt gegarandeerd door replicatie, waardoor we snel kunnen overschakelen als er een probleem is in onze database.
Binnen ArcelorMittal wordt gebruik gemaakt van de azure omgeving van Microsoft, waardoor CosmosDB een logische keuze is voor ons project.
CosmosDB ondersteunt verschillende API's zoals SQL, MongoDB, Cassandra, Gremlin en Table API, waardoor we flexibel kunnen werken met verschillende soorten data.
De connectie met CosmosDB gebeurt via een private endpoint, waardoor we de data veilig kunnen opslaan en allen kunnen raadplegen via het domein van op ArcelorMittal Gent.
\subsection{Gremlin API}
Gremlin is een database query taal die gebruikt wordt om te communiceren met grafiek databases zoals CosmosDB \autocite{Tinkerpop2023}.\@
De taal bevat verschillende varianten zoals Gremlin-Java, Gremlin-Python en Gremlin-Groovy,\dots
In ons geval zullen we Gremlin-Javascript gebruiken om de data van ArcelorMittal Gent in CosmosDB te bevragen. De bevraging is gebasseerd op een RestAPI die de data ophaalt en teruggeeft in JSON formaat.
Javascript is hiervoor geschikt omdat Javascript en JSON een goede combinatie zijn om data te verwerken, daarnaast maken we ook gebruik van NodeJS wat met JavaScript werkt. 
Als extra kan je met JavaScript eenvoudig front end en back end combineren, wat mogelijk is voor verdere uitbereiding van deze thesis.
Hiernaast hebben we ook Neo4J overwogen met cypher als query taal, maar deze heeft zijn eigen ecosysteem en is niet even flexibel en schaalbaar als Gremlin API.\@
Gremlin daarentegen voorziet dat alle databases die TinkerPop-enabled zijn, kunnen worden gebruikt. Hieronder vallen onder andere Amazon Neptune, CosmosDB, JanusGraph en nog vele andere.\autocite{Tinkerpop2023a}

\subsection{NodeJS}
NodeJS is een open-source JavaScript runtime-omgeving die de mogelijkheid biedt om JavaScript-code uit te voeren op de server ~\autocite{NodeJS2022}.
Dit gebeurt via een Single-Threaded, Non-Blocking I/O model wat betekent dat er geen nieuwe threads worden aangemaakt voor elke request.
Door middel van Callbacks en Promises werkt NodeJS asynchroon, wat betekent dat de code niet wacht op een antwoord van een request maar ondertussen andere requests kan verwerken.
Met event loops worden de requests in een wachtrij geplaatst en worden ze verwerkt wanneer de server klaar is met een andere operatie.
Daardoor is NodeJS zeer performant en schaalbaar voor het verwerken van grote hoeveelheden data.

De technologie is in 2009 ontwikkeld en geïntroceerd door Ryan Dahl en is sindsdien zeer populair geworden in de webontwikkeling. 
NodeJS werd later door grote bedrijven zoals Netflix, eBay \& Uber gebruikt voor hun back-end systemen.
Zoals eerder vermeld is het geen framework maar een runtime-omgeving, dit betekent dat het geen standaardbibliotheken heeft die hergebruikt kunnen worden door developers.
Dit biedt volledig vrijheid in hoe de architectuur van de applicatie wordt opbouwt.

Een van de nadelen van NodeJS is dat het single-threaded is, wat betekent dat het niet geschikt is voor CPU-intensieve taken.
Dit komt omdat NodeJS gebruik maakt van een event loop die de requests in een wachtrij plaatst en ze verwerkt wanneer de server klaar is met een andere operatie.
Hierdoor kan het zijn dat een request die veel tijd nodig heeft om te verwerken de andere requests blokkeert.
Sinds 2018 is er een nieuwe feature geïntroduceerd in NodeJS genaamd Worker Threads, dit maakt het mogelijk om multi-threaded te werken.
Daardoor kunnen we de CPU-intensieve taken in een aparte thread verwerken en de main thread vrij houden voor andere requests.

\subsection{EPCIS-events}
Voor dit onderzoek maken we gebruik van EPCIS (Electronic Product Code Information Services), dit is een GS1-standaard die bedrijven in staat stelt om gebeurtenissen in de toeleveringsketen vast te leggen en te delen. 
Het biedt een gemeenschappelijk kader voor het vastleggen van de wat, wanneer, waar en waarom van gebeurtenissen die betrekking hebben op fysieke of digitale objecten. 
Deze waarden zijn ontwikkeld door GS1 om gegevens over beweging, status en verandering van een item in de toeleveringsketen (supply chain) vast te leggen en te delen binnen en buiten het bedrijf \autocite{Devins}.
``Met behulp van deze waarden kunnen we real-life objecten omzetten in elektronisch opgeslagen informatie, waarna we dit kunnen communiceren met eindgebruikers.`` zegt \textcite{Devins}.
Door deze normen toe te passen kunnen we de traceerbaarheid van het product per proces garanderen inclusief de gewenste parameters die opgeslagen worden in ons grafiekmodel zoals tijd (wanneer) en temperatuur (hoe), waar nodig.
Voor deze relaties goed en volgens de normen op te stellen, maken we gebruik van de EPCIS-events. Dit is een referentielijst waarin verschillende acties vooraf zijn bepaald.
Hier hebben we bijvoorbeeld acties zoals ``add'', ``delete'' en ``update'' die we gebruiken om de data te structureren.
De add-actie wordt gebruikt om een nieuwe knoop toe te voegen aan het grafiekmodel, de delete-actie om een knoop te verwijderen en de update-actie om een knoop bij te werken.
Bij deze actie voegen we ook properties toe in een event-lijst, daar bepalen we welk soort event we gebruiken en voegen we eventueel extra parameters toe voor op de relatie, zoals tijd.
In figuur~\ref{fig:jsonld} is een voorbeeld te zien van een associatie-event tussen een machine en een component.

De belangrijkste voordelen volgens \textcite{GS12025} staan opgelijst in tabel~\ref{tab:epcis-voordelen}.
\begin{table}[H]
    \centering
     \begin{tabular}{lp{0.6\textwidth}}
          \toprule
          \textbf{Voordeel} & \textbf{Beschrijving} \\
          \toprule
          Verbeterde zichtbaarheid & Door het vastleggen en delen van gedetailleerde gebeurtenisgegevens kunnen bedrijven beter inzicht krijgen in de bewegingen en status van producten in de toeleveringsketen. \\
          \midrule
          Efficiëntieverbeteringen & Door het automatiseren van gegevensverzameling en -uitwisseling kunnen bedrijven operationele efficiëntie verbeteren en fouten verminderen. \\
          \midrule
          Naleving van regelgeving & EPCIS helpt bedrijven te voldoen aan wettelijke vereisten voor traceerbaarheid en rapportage. \\
          \midrule
          Betere samenwerking & Door het delen van gebeurtenisgegevens met partners kunnen bedrijven beter samenwerken en de toeleveringsketen optimaliseren. \\
          \bottomrule
     \end{tabular}
     \caption[Belangrijkste voordelen van EPCIS volgens GS1]{\label{tab:epcis-voordelen}}
\end{table}

\subsection{GS1}
GS1 is een wereldwijde organisatie die standaarden ontwikkelt voor identificatie, codering \& uitwisseling~\autocite{GS1standards}.
Dit zijn standaarden die bedrijven helpen om hun producten en diensten te identificeren, traceren en uit te wisselen.
Elk product heeft een unieke identificatiecode die het mogelijk maakt om het product te traceren doorheen de toeleveringsketen.
GS1 heeft verschillende standaarden ontwikkeld zoals de GTIN-code (Global Trade Item Number), GLN-code (Global Location Number) en SSCC-code (Serial Shipping Container Code).
De GTIN-code is een code voor producten, elk product dat traceerbaar wil zijn moet een GTIN-code hebben. 
Indien de GTIN niet aanwezig is kunnen er verschillende stappen ondernomen worden, in figuur~\ref{fig:gtin} is een overzicht te zien van de verschillende stappen die ondernomen kunnen worden.
Ook locaties kunnen een unieke code krijgen, dit is dan de GLN-code. In volgende secties gaan we dieper in op de ontwikkeling van de GTIN-code en de GLN-code, de rest van de GS1-standaarden zijn niet relevant voor dit onderzoek.

\subsubsection{GTIN}
De GTIN-code ofwel het Global Trade Item Number is een unieke identificatiecode die wordt gebruikt om producten te identificeren in de toeleveringsketen \autocite{GTIN2025}.
Deze bestaat uit 7 tot 11 cijfers benoemd met GTIN-8, GTIN-13 en GTIN-14, de groote van de code hangt af van de toepassing en het type product.
Producten die klein zijn en weinig informatie bevatten kunnen een GTIN-8 hebben, terwijl grotere producten zoals dozen of pallets een GTIN-14 kunnen hebben.
In de gezondheidszorg wordt vaak een GTIN-14 ook toegelaten aangezien er ook veel informatie nodig kan zijn voor bijvoorbeeld medicatie.
De opbouw van de code is vrij simpel: de eerste 7 tot 11 cijfers identificeren het bedrijf, hoe korter deze prefix is, hoe meer producten er geïdentificeerd kunnen worden.
Na deze code volgt de productcode, dit is ook een reeks unieke cijfers voor identificatie van het product.
Als laatste volgt het controlecijfer, dit is een cijfer dat wordt berekend op basis van de andere cijfers in de code.
Dit cijfer wordt gebruikt om te controleren of de code correct is ingevoerd en of er geen fouten zijn gemaakt bij het scannen van de code.

\begin{figure}[h]
     \centering
     \includegraphics[width=0.8\textwidth]{./img/GTIN.png}
     \caption[GTIN stappenplan]{\label{fig:gtin} GTIN stappenplan. Geraadpleegd op~\cite{GTIN2025} }
\end{figure}

\subsubsection{GLN}
De GLN-code ofwel het Global Location Number is een unieke identificatiecode die wordt gebruikt om locaties te identificeren in de toeleveringsketen \autocite{GLN}.
Deze code is vergelijkbaar met de GTIN-code maar wordt gebruikt voor locaties, de opbouw daarentegen is gelijkaardig.
De GLN code moet sinds 1 juli 2022 een uniek cijfer zijn en mag niet hergebruikt worden voor andere locaties.
De opbouw is zoals eerder vermeld gelijkaardig, eerst hebben we de bedrijfsprefix, daarna volgt het adresnummer en als laatste een controlecijfer.
De bedrijfsprefix is een unieke code die wordt toegekend aan het bedrijf, deze code kan aanvraagd worden bij GS1. Het adresnummer is een unieke code die wordt toegekend aan de locatie, dit kan bijvoorbeeld een gebouw of een afdeling zijn.

\subsection{Schema.org}
Schema.org is een grote verzameling van gestructureerde data die entiteiten (knopen) en relaties kan weergeven~\autocite{Douglas2023}.
Schema.org is te vergelijken met GS1 maar dan meer generiek. In ons project zullen we de combinatie van schema.org objecten en GS1 objecten gebruiken om de data te structureren.
Om een voorbeeld te geven hebben we in de data gebruik gemaakt van een splitsing tussen locaties en assets.
Elke locatie krijgt een type, een label en extra properties zoals een adres of een geografische locatie.
Dit type of ID kan bijvoorbeeld ``Place'' zijn voor een locatie of ``Product'' voor een asset. 
Op de website van schema.org zijn alle mogelijkheden beschikbaar om de data te structureren en te annoteren.
Daardoor houden we een duidelijk onderscheid tussen de soorten afdelingen en machines binnen de afdeling.
In codefragment~\ref{fig:jsonld} is een voorbeeld te zien van een JSON-LD bestand met gegevens volgens schema.org in het eerste json-object.
Voor de locatie van Gent hebben we ID 123, is het type een plaats en kunnen we de vorige knoop teruggeven met containedInPlace.
Verder kunnen properties toegevoegd worden zoals eronder aangegeven met een key-value paar.
% In het codeblok \ref{lst:jsonld} is een voorbeeld te zien van een JSON-LD bestand met gegevens.
\begin{listing}
     \begin{minted}{jsonld}
          {
          "@context": {
               "schema":"https://schema.org",
               "epcis": "https://ref.gs1.org/epcis/",
               "cbv": "https://ref.gs1.org/cbv/"
          }
               "@graph": [
                    {
                         "@id": "123",
                         "@type": "Place",
                         "name": "Gent",
                         "label": "Gent",
                         "schema:containedInPlace": 321,
                         "KEY": "VALUE"
                    },
                    {
                         "@type": "epcis:EPCISDocument",
                         "schemaVersion": "2,0",
                         "creationDate": "2025-03-19T17:40:54Z",
                         "epcis:EPCISBody": {
                              "epcis:EventList": {
                                   "epcis:AssociationEvent": {
                                   "eventTime": "2025-03-19T17:40:54Z",
                                   "eventTimeZoneOffset": "+01:00",
                                   "parentID": "[Machine 1]",
                                   "childEPCs": [
                                        "[Motor 1]"
                                   ],
                                   "action": "ADD",
                                   "disposition": "cbv:disp:active"
                                   }
                              }
                         }
                    }
               ]
          }
     \end{minted}
     \caption[Voorbeeld JSON-LD bestand]{\label{fig:jsonld}Voorbeeld van een JSON-LD bestand met locatiegegevens.}
\end{listing}

\section{Data preprocessing}
De data die we gebruiken komt uit een SAP systeem. Deze data bevat een functionele boom structuur van de verschillende levels binnen ArcelorMittal Gent.
Dit houdt in dat de data in een hiërarchische structuur is opgebouwd, waarbij de verschillende processen met elkaar verbonden zijn.
Voor het ontcijferen van deze data hebben we verschillende specialisten binnen ArcelorMittal gecontacteerd die ons hebben geholpen met het vertalen van de datacodes.
Alleen voor deze opsplitsing tussen locatie en machine is er een grote moeilijkheid. Normalieter zou deze SAP data opgeplitst zijn zodat level 1 en level 2 de afdelingen zijn en vanaf level 3 machines. 
Helaas is in dit onderzoek geen garantie dat dit klopt, na controle lijkt dit voor 80\% van de data te kloppen, maar als een afdeling uitgebreider is kan dit verschillen.
Doordat de data enorm groot is waardoor handmatig opsplitsen eindeloos werk is en er geen garantie is op deze splitsing gaan we er vanuit dat het wel klopt dat level 2 de splitsing is. 
Dit betekent dat er voor verdere integratie een structurele aanpassing moet gebeuren aan de dataset.
Zo gaan we er van uit dat level 1 in de boom de site zelf is en dat level 2 de verschillende afdelingen zijn binnen de site. Zo gaat het telkens dieper tot level 12 waar de verschillende onderdelen van de machines zich bevinden.
Elke machine heeft een unieke code die we uit een dictionairy kunnen halen die we hebben ontvangen van ArcelorMittal.
Deze ontcijferingen zijn nodig in een later proces om onze chatbot een correct antwoord te laten geven op de vragen van de gebruiker.

\subsection{JSON tot Graph}
De data die we ontvangen van ArcelorMittal uit SAP is in JSON formaat. Deze data stelt ons in staat om door de beschrijving van de sleutelwoorden de data te ontcijferen en te verwerken in ons grafiekmodel.
Omdat wij voor Gremlin en EPCIS werken met schema.org hebben wij een JSON-LD formaat nodig. Dit is een uitbreiding van JSON die het mogelijk maakt om data te structureren en te annoteren.
Daarna kunnen we deze JSON-LD in een folder plaatsen en wordt deze automatisch ingelezen met een NodeJS script.
Deze gaat voor alle types een knoop aanmaken en de bestaande properties toevoegen aan de JSON-LD.\@ Hierbij kunnen we nog bepalen welke properties we gebruiken en zelf de key van de properties bepalen.
Daarna kunnen we ook de relaties aanmaken volgens de schema.org structuur, dit gebeurt door de huidige knoop ID te verbinden met het ID van de parent, deze zitten ook in de SAP data verworven en kunnen dus makkelijk eruit gehaald worden.
Om de relaties te bepalen maken we gebruik van het hierarchisch level. Deze property bepaald of het een locatie is, een machine of onderdelen van de machine, in onze terminologie noemen we dit ook wel de assets.
Door de combinatie van het level en de parent kunnen we zelf bepalen in de vertaling van JSON naar JSON-LD welke relatie op welk level gebruikt kan worden.

\subsubsection{JSON-LD}
JSON-LD staat voor JavaScript Object Notation for Linked Data. Dit is een uitbereiding van JSON die het mogelijk maakt om data te structureren en te annoteren~\autocite{jsonld.org}.
Het doel van deze Linked Data is om te zorgen dat je kan beginnen bij één deel en van daaruit de embedded links kan volgen. 
Dit doet je vast denken aan ons grafiekmodel waarbij je ook kan beginnen bij een knoop en van daaruit de verschillende relaties kan volgen.
Het principe is hetzelfde, alleen wordt het model hier gecodeerd in code die makkelijk te lezen en te schrijven is voor mensen.
In samenwerking met schema.org en GS1 kunnen we de data structureren, normeren en linken aan elkaar, daarom is dit zeer geschikt voor ons project. 
Daarna kunnen we deze JSON-LD direct inlezen in CosmosDB om deze linken op te slaan en te visualiseren.
Er bestaan ook andere formaten zoals PYLD, dotNetRDF, \dots, maar doordat onze omgeving in JavaScript is opgebouwd is JSON-LD een logische keuze.
Daarnaast zijn veel mensen bekend met JSON, waardoor implementatie en gebruik eenvoudiger zijn.

\begin{figure}[h]
     \centering
     \includegraphics[width=0.8\textwidth]{./img/grapmodel_example.png}
     \caption[Voorbeeld Grafiekmodel.]{\label{fig:graphmodel}Voorbeeld van kleinschalig grafiekmodel, gemaakt met mock data.}
\end{figure}

\section{Chatbot}
De chatbot is een belangrijk onderdeel van ons project, aangezien we zonder deze chatbot onze data moeilijk kunnen doorzoeken.
Hij zal ons helpen om de data snel en efficiënt te doorzoeken en de juiste informatie te vinden.
In deze thesis is de chatbot een API die de vragen van de gebruiker kan beantwoorden en advies kan geven op basis van de data die we hebben verzameld.
In dit deel maken we ook gebruik van één python script die de data en het AI model gaat laten samenwerken.
We maken hier gebruik van Python omdat dit een heel toegankelijke taal is voor het opzetten van AI modellen en verwerken van data.
Via onze NodeJS runtime omgeving kunnen we deze python script aanroepen en de data doorsturen naar de chatbot.

\subsection{Ollama}
Voor ons lokaal model maken we gebruik van Ollama. Dit is een open-source bibliotheek die verschillende vooraf getrainde modellen bevat. 
Dat is een groot voordeel omdat we in deze thesis de scope niet leggen op het maken van een large language model die natuurlijke taal kan verwerken.
Wel zijn we op zoek moeten gaan naar het beste model voor het maken van Gremlin queries. Dit is geen eenvoudige klus, dit omdat er verschillende Gremlin versies bestaan die een andere query syntax hebben.
Dit kan dus resulteren in queries die niet uitvoerbaar zijn in onze CosmosDB.\@
Tijdens de testfase zijn er verschillende modellen getest geweest, zoals llama2, llama3 \& Phi4.
De llama modellen 2 en 3 waren heel goed in het omzetten van de uitvoer naar natuurlijke taal, maar hadden veel moeite met het omzetten van de input naar een Gremlin query.
Daarna hebben we ook gekeken naar Phi4 die met de benodigde finetuning een goede output gaf voor de Gremlin queries en ook de natuurlijke taal goed kon omzetten.
In het volgende hoofdstuk gaan we iets dieper in op de verschillende modellen en hun resultaten.

\subsubsection{llama2}
Llama 2 is een open-source large language model ontwikkeld door Meta AI~\autocite{llama2}. 
Het model is beschikbaar in verschillende versies, variërend van 7 tot 70 miljard parameters, afhankelijk van de gekozen grootte.
Llama 2 is ontworpen om te concurreren met andere geavanceerde modellen zoals GPT-3.5 en GPT-4.
De training van het model is gebaseerd op een combinatie van supervised fine-tuning en reinforcement learning met menselijke feedback (RLHF).
Echter blijven er altijd uitdagingen zoals hallucinaties en bias die net als andere LLM's kunnen blijven optreden.

\subsubsection{llama3}
Llama 3 is de opvolger van Llama 2 en is een state-of-the-art large language model dat is ontwikkeld door Meta AI~\autocite{Meta2024}.
state-of-the-art wil zeggen dat het model de nieuwste technieken en algoritmen gebruikt om de prestaties te verbeteren.
Llama 3 heeft een aantal verbeteringen ten opzichte van zijn voorganger llama 2, waaronder een grotere dataset en een verbeterde architectuur.
Het model is ook, net als zijn voorganger, beschikbaar in verschillende versies. Variërend van 7 tot 70 miljard parameters, afhankelijk van de gekozen grootte.
Llama 3 heeft ook iets verbeterde codeereigenschappen, maar in testen bleek dat het nog steeds moeite had met het genereren van Gremlin queries.

\subsubsection{Phi4}
Phi4 is een state-of-the-art large language model dat is ontwikkeld door Microsoft en heeft 14B parameters~\autocite{Kamar2024}.
Het doel van Phi4 is dat het een compact, maar krachtig en snel model is dat kan concurreren met andere grote modellen zoals llama3 zoals hiervoor besproken.
De testen met phi 4 tegenover andere grote modellen toonden aan dat het model zeer goed presteerd in vergelijking met andere modellen.
In figuur~\ref{fig:MMLU} zien we de scores van de MMLU benchmark, dit is een benchmark die verschillende onderwerpen aanpakt zoals coderen, wiskunde en taal.
Hier kunnen we afleiden dat Phi4 een hoge score heeft, maar zeer gelijkaardig loopt met het grootste model van Llama 3.
Dus phi4 is in ons geval een goede keuze omdat het een lichtgewicht model is en ook goed presteert in ons scenario, mits finetuning met context.

\begin{figure}[h]
     \centering
     \includegraphics[width=0.8\textwidth]{./img/MMLU.png}
     \caption[Voorbeeld Grafiekmodel.]{\label{fig:MMLU}Voorbeeld van kleinschalig grafiekmodel, gemaakt met mock data.}
\end{figure}

\subsection{Retrieval Augmented Generation}
Om onze chatbot te optimaliseren gaan we gebruik maken van RAG.\@
Dit is een techniek die het mogelijk maakt om de chatbot te laten leren van de data die we hebben verzameld in ongestructureerde teksten, databases of andere bronnen~\autocite{Zeichick2023}.
Hierbij hebben we een simpel tekstbestand aangemaakt waarbij de nodige informatie meegegeven wordt.
Zo hebben we de context voor voor het maken van de Gremlin queries, daarbij geven we een aantal queries mee die zeker werken.
Dit doen we omdat het model soms moeite heeft met de juiste versie van Gremlin te gebruiken.
Daarnaast is het belangrijk dat we als context meegeven dat hij geen tekst of codeblokken mag genereren maar enkel en alleen de query mag teruggeven.
Dit is belangrijk omdat we deze query direct implementeren in de database, wat betekent dat als er overige tekst of karakters in staan, dat het genereren zal mislukken door syntax fouten.
Naast de context voor het genereren van de query hebben we ook context voor de chatbot die in natuurlijke taal zal antwoorden.
Het is belangrijk dat hij de JSON met items kan omzetten naar natuurlijke taal en begrijpt wat er wel of niet meegegeven mag worden.
Als voorbeeld hebben we dat er in het antwoord soms database performantiemetingen meegegeven worden en deze zijn niet relevant voor de gebruiker.
%%=============================================================================
%% Methodologie
%%=============================================================================

\chapter{\IfLanguageName{dutch}{Methodologie}{Methodology}}%
\label{ch:methodologie}
\section{Dataverwerking}
In het eerste deel van deze bachelorproef wordt er een literatuurstudie uitgevoerd over de verschillende technologieën die we gebruiken.
Hier hebben we de werking en efficiëntie van CosmosDB onderzocht. Daarnaast hebben we ook gekeken naar de werking van Gremlin API en hoe we deze kunnen gebruiken om een grafiekmodel op te zetten.
Ook is het belangrijk om te weten dat we deze data hebben omgezet naar een jsonLD-bestand die volgens EPCIS normen is opgesteld. Dit garandeerd de duidelijkheid van de data.

\section{Proof of concept}
\subsection{Opzetten van een grafiekmodel}
In het tweede deel van deze bachelorproef wordt er een proof of concept opgesteld van het grafiekmodel waarbij we de data van ArcelorMittal Gent zullen gebruiken.
Zoals eerder vermeld hebben we de data omgezet naar EPCIS waarden in een jsonLD-bestand. Dit bestand wordt ingeladen in CosmosDB via nodejs en Gremlin API.\@
Daardoor onstaat een grafiekmodel waarin we de data kunnen visualiseren en analyseren door de relaties tussen de verschillende producten en processen.
Dit grafiekmodel kan gevisualiseerd worden met behulp van Graph Explorer of andere visualisatie tools.

\subsection{Opzetten van een chatbot}
In het derde deel van deze bachelorproef wordt er een chatbot opgesteld die de data van het grafiekmodel kan bevragen en analyseren.
De input van dit model is een vraag die de gebruiker stelt, deze vraag wordt vertaald naar een Gremlin query die de data in CosmosDB zal bevragen.
Die kan dan op zijn beurt de relevante knopen en edges vinden, die worden verwerkt en geformatteerd in leesbare tekst.
%% TODO: In dit hoofstuk geef je een korte toelichting over hoe je te werk bent
%% gegaan. Verdeel je onderzoek in grote fasen, en licht in elke fase toe wat
%% de doelstelling was, welke deliverables daar uit gekomen zijn, en welke
%% onderzoeksmethoden je daarbij toegepast hebt. Verantwoord waarom je
%% op deze manier te werk gegaan bent.
%% 
%% Voorbeelden van zulke fasen zijn: literatuurstudie, opstellen van een
%% requirements-analyse, opstellen long-list (bij vergelijkende studie),
%% selectie van geschikte tools (bij vergelijkende studie, "short-list"),
%% opzetten testopstelling/PoC, uitvoeren testen en verzamelen
%% van resultaten, analyse van resultaten, ...
%%
%% !!!!! LET OP !!!!!
%%
%% Het is uitdrukkelijk NIET de bedoeling dat je het grootste deel van de corpus
%% van je bachelorproef in dit hoofstuk verwerkt! Dit hoofdstuk is eerder een
%% kort overzicht van je plan van aanpak.
%%
%% Maak voor elke fase (behalve het literatuuronderzoek) een NIEUW HOOFDSTUK aan
%% en geef het een gepaste titel.





% Voeg hier je eigen hoofdstukken toe die de ``corpus'' van je bachelorproef
% vormen. De structuur en titels hangen af van je eigen onderzoek. Je kan bv.
% elke fase in je onderzoek in een apart hoofdstuk bespreken.

%\input{...}
%\input{...}
%...

%%=============================================================================
%% Conclusie
%%=============================================================================

\chapter{Conclusie}%
\label{ch:conclusie}
In deze bachelorproef is onderzocht hoe een combinatie van een grafiekmodel en een Large Language Model (LLM) kan worden ingezet om gegevens binnen het staalverwerkingsproces van ArcelorMittal Gent efficiënt op te vragen.
Het doel was een proof of concept te ontwikkelen die niet alleen de traceerbaarheid van productieprocessen verbetert, maar ook de operationele efficiëntie verhoogt door middel van een chatbot die procesvragen kan beantwoorden.
Door de SAP-boomstructuur om te zetten naar JSON-LD-formaat volgens GS1-standaarden en deze te structureren in een grafiekdatabase (Cosmos DB), werd een stevige basis gelegd voor het traceren van productieprocessen.
Vervolgens werd een chatbot ontwikkeld die gebruikmaakt van Gremlin-queries en Retrieval Augmented Generation (RAG) om procesvragen te vertalen naar begrijpelijke antwoorden.
Met modellen zoals Code Llama voor querygeneratie en Phi-4 voor natuurlijke taalverwerking is een performante oplossing opgezet die uitbreidbaar is en bruikbaar voor andere use-cases binnen de industrie.

In dit onderzoek is vooral gezocht naar combinaties van technieken die de traceerbaarheid van productieprocessen verbeteren.
De ontwikkelde proof of concept toont aan dat het mogelijk is een chatbot te creëren die vragen kan beantwoorden over productieprocessen door gebruik te maken van dit grafiekmodel en een LLM.\@
Met de juiste finetuning en eventuele hertraining van de modellen kan deze oplossing verder worden uitgebreid en geoptimaliseerd voor andere toepassingen binnen de industrie.
Naast de technische aspecten is het belangrijk de gebruikerservaring mee te nemen.
De chatbot functioneert als een API die geïntegreerd kan worden in bestaande systemen, waardoor gebruikers eenvoudig toegang krijgen tot de benodigde informatie.
De proof of concept biedt een solide basis voor verdere ontwikkeling en implementatie binnen ArcelorMittal Gent en mogelijk ook andere bedrijven in de staalindustrie.

\section{Resultaten}
De resultaten van dit eerste onderzoek zijn veelbelovend. Er is aangetoond dat het mogelijk is om een grafiekmodel op te zetten met behulp van Cosmos DB en de Gremlin API, en dat dit model kan worden gebruikt om data te visualiseren en te analyseren.
Daarnaast is een proof of concept ontwikkeld van een chatbot die vragen kan beantwoorden over het productieproces van ArcelorMittal Gent.
Een belangrijk knelpunt is dat het CodeLlama-model niet altijd de juiste Gremlin-query genereert, waardoor handmatig queries toegevoegd moeten worden aan het JSON-bestand voor Elasticsearch.
Dit kan opgelost worden door het model verder te trainen met LoRA (zoals besproken in hoofdstuk~\ref{sec:LORA}), maar vanwege beperkte tijd en resources is dit niet uitgevoerd in deze bachelorproef.

In de volgende \href{https://youtu.be/D4-bSRYDLWM}{demo video} is een korte demonstratie te zien van de chatbot en hoe deze werkt met de data die in Cosmos DB is opgezet.\@
Aan het begin heeft het Ollama-model tijd nodig om een vraag te beantwoorden, omdat Ollama standby staat maar het model nog niet geladen is.
Pas zodra de vraag wordt gesteld, wordt het model geladen en kan de chatbot de vraag beantwoorden.
Bij een tweede vraag verloopt dit veel sneller, omdat het model dan al geladen is en de chatbot direct kan antwoorden.
In figuur~\ref{fig:demo} is een screenshot te zien van de demo video.

\subsection{Voordelen}
De voordelen van het gebruik van een tijdelijk grafiekmodel zijn onder andere:
\begin{itemize}
    \item Het is mogelijk om data te visualiseren en analyseren door de relaties tussen de verschillende producten en processen.
    \item Het is mogelijk om snel vragen van gebruikers te beantwoorden over het productieproces van ArcelorMittal Gent.
    \item Verhoogde traceerbaarheid van productieprocessen, waardoor het eenvoudiger wordt om problemen te identificeren en op te lossen.
    \item Het is mogelijk om de chatbot eenvoudig op te zetten en te beheren, zonder zorgen over de onderliggende infrastructuur.
    \item Door gebruik te maken van Elasticsearch kan er sneller een antwoord geven op meest gestelde vragen, waardoor de chatbot sneller en efficiënter wordt.
\end{itemize}

\subsection{Nadelen}
De nadelen of problemen die ondervonden zijn bij de huidige werkwijze zijn onder andere:
\begin{itemize}
    \item Het CodeLlama model genereert niet altijd de juiste Gremlin query, waardoor ze handmatig moeten worden toegevoegd aan het JSON bestand voor Elasticsearch.
    \item De snelheid van de chatbot is afhankelijk van Ollama die een antwoord of query moet genereren, waardoor het soms langer duurt om een antwoord te krijgen indien de query niet in Elasticsearch aanwezig is.
\end{itemize}

\section{Toekomstig onderzoek}
Om de chatbot verder uit te breiden en te verbeteren, kunnen verschillende zaken worden aangepakt.
Allereerst kan gekeken worden naar het finetunen van het CodeLlama-model met LoRA, zoals besproken in~\ref{sec:LORA}, om de kwaliteit van de gegenereerde Gremlin-queries te verbeteren.
Daarnaast kunnen er nog meer prestatieoptimalisaties worden doorgevoerd om de snelheid van de chatbot te verhogen.
Ook kunnen rechten worden toegevoegd, zodat bepaalde gebruikersgroepen alleen specifieke vragen kunnen stellen of antwoorden kunnen ontvangen.
Daarnaast kunnen deze rechten worden gebruikt om DELETE-, ADD- en UPDATE-acties toe te voegen, waardoor de chatbot multifunctioneel wordt.
% TODO: Trek een duidelijke conclusie, in de vorm van een antwoord op de
% onderzoeksvra(a)g(en). Wat was jouw bijdrage aan het onderzoeksdomein en
% hoe biedt dit meerwaarde aan het vakgebied/doelgroep? 
% Reflecteer kritisch over het resultaat. In Engelse teksten wordt deze sectie
% ``Discussion'' genoemd. Had je deze uitkomst verwacht? Zijn er zaken die nog
% niet duidelijk zijn?
% Heeft het onderzoek geleid tot nieuwe vragen die uitnodigen tot verder 
%onderzoek?




%---------- Bijlagen -----------------------------------------------------------

\appendix

\chapter{Onderzoeksvoorstel}

Het onderwerp van deze bachelorproef is gebaseerd op een onderzoeksvoorstel dat vooraf werd beoordeeld door de promotor. Dat voorstel is opgenomen in deze bijlage.

%% TODO: 
\section*{Samenvatting}
    In deze bachelorproef onderzoeken we hoe grafiekmodellering kan bijdragen aan efficiënte klachtenafhandeling binnen bedrijfsprocessen. 
    Door gebruik te maken van een large language model (LLM) gekoppeld aan een chatbot willen we bedrijven in staat stellen sneller en nauwkeuriger de oorzaken van problemen te achterhalen, waardoor kostbaar handmatig werk wordt verminderd. 
    Dit door een vraag te stellen aan de chatbot waarbij hij via het grafiekmodel een ongewone gebeurtenis kan ophalen. Het onderzoek richt zich op twee fases: de omzetting van EPCIS-events naar een grafiekmodel in Cosmos DB met behulp van de Gremlin API, en het trainen van een LLM om te analyseren waar, wanneer, wat en hoe gebeurtenissen plaatsvinden. 
    Dit proces helpt bij het opsporen van anomalieën, met als doel schaalbare en performante oplossingen te bieden voor bedrijfs- en nationale use cases. 
    Uiteindelijk streven we naar het optimaliseren van klantenervaring en operationele efficiëntie.

% Kopieer en plak hier de samenvatting (abstract) van je onderzoeksvoorstel.

% Verwijzing naar het bestand met de inhoud van het onderzoeksvoorstel
%---------- Inleiding ---------------------------------------------------------

% TODO: Is dit voorstel gebaseerd op een paper van Research Methods die je
% vorig jaar hebt ingediend? Heb je daarbij eventueel samengewerkt met een
% andere student?
% Zo ja, haal dan de tekst hieronder uit commentaar en pas aan.

%\paragraph{Opmerking}

% Dit voorstel is gebaseerd op het onderzoeksvoorstel dat werd geschreven in het
% kader van het vak Research Methods dat ik (vorig/dit) academiejaar heb
% uitgewerkt (met medesturent VOORNAAM NAAM als mede-auteur).
% 

\section{Inleiding}%
\label{sec:inleiding}
Binnen de Arcelor Mittal Gent bestaan er veel verschillende processen die verspreid zijn over meerdere afdelingen.
Hierbij bevat elke afdeling een deel van het proces om van grondstof tot een afgewerkt product te komen. 
Tussen elke afdeling bevindt zich een deel van de toeleveringsketen, waarbij het resultaat van de ene afdeling het beginpunt is voor de volgende afdeling.
Dit zorgt ervoor dat iedere stap zijn invloed heeft op de kwaliteit van het eindresultaat.

Elke afdeling is opgebouwd rond een specifiek proces, zoals de hoogoven die ruwe grondstoffen omzet in ruw ijzer en de staalfabriek die ruw ijzer in staal verandert.
Omdat deze afdelingen zeer specifieke processen hebben, hebben ze ook zeer specifieke behoeften en datamodellen.
Dit maakt analyse of onderzoek over afdelingen heen zeer complex en tijdsrovend.

De scope van deze bachelorproef is tweedelig, waarbij deel één eruit bestaat om samen met Tracked vanuit een gestandariseerd datamodel die zichtbaarheid events bijhoudt om te zetten naar een grafiekmodel.
Deel twee bouwt verder voort op dit grafiekmodel om dan een LLM (large language model) te trainen die eenvoudig informatie eruit kan halen en teruggeven.

De centrale vraag die hier gesteld wordt is: ``Hoe we efficiënt en snel grafiekmodellering kunnen toepassen om een LLM te ontwikkelen die in staat is om het waar, wanneer, wat en hoe van gebeurtenissen binnen een proces vast te stellen, ter ondersteuning van klachtenafhandeling bij productfouten.``
%---------- Stand van zaken ---------------------------------------------------

\section{Literatuurstudie}%
\label{sec:literatuurstudie}

\subsection{Cosmos DB}%
\subsubsection{Waarom Cosmos DB gebruiken}
Cosmos DB is een NoSQL-database van microsoft, deze heeft een lage latentie, multi-query-api die makkelijk grote hoeveelheden data kan verwerken en heeft een grote beschikbaarheid zegt~\textcite{Put2020} wat zeer belangrijk is in ons project.
Naast deze feiten is CosmosDB horizontaal schaalbaar wat betekent dat we op hoogtepunten een miljoen lees- en schrijfaanvragen kunnen verwerken door het nodige aantal servers toe te voegen.
De hoge beschikbaarheid wordt gegarandeerd door replicatie waardoor we snel kunnen overschakelen als er iets fout gaat in de database.

\subsection{Grafiek modellering}
Cosmos DB biedt ook grafiek modellering aan met behulp van Gremlin. Gremlin is een grafiek database service waarbij we grote grafieken kunnen opslaan die bestaan uit vertices en edges \autocite{Microsoft2024}.
Dit gebeurt met een snelheid van miliseconden waardoor we een snelle verwerkingstijd zullen kunnen neerzetten.
Ook is Gremlin schaalbaar en kan de consistentie level gekozen worden om een balans te vinden tussen consistentie, beschikbaarheid en latentie.
Binnen zo een grafiek model hebben we Vertices (of nodes) die een persoon, plaats of event beschrijven zoals in ons geval bijvoorbeeld een slab die verplaatst wordt van A naar B.

\subsubsection{Verschil transformatie en aggregatie}
Binnen ons project kunnen bestaansvormen van de producten veranderen. Zo hebben we transformatie waarbij de cokes bijvoorbeeld omgezet worden in vloeibaar staal, dit is niet omkeerbaar.
Bij aggregatie kunnen we terug naar de oorspronkelijke bestaandsvorm zoals bijvoorbeeld van warmwalserij naar koudwalserij waarbij de platen dunner worden en eventueel een roestvrije laag krijgen \autocite{ArcelorMittal2024}.
In onze grafiek modellering moeten we gebruik maken van tijdelijke edges om ook terug te kunnen kijken naar wat voor het transformatie- of aggregatieproces gebeurd is om anomalieën terug te vinden \autocite{JaewookByun2020}.


\subsection{EPCIS}
Voor dit onderzoek moet alles voldoen aan de EPCIS (Electronic Product Code Information Services) waarden, dit zijn de wat, wanneer, waar, waarom en hoe. 
Deze waarden zijn ontwikkeld door GS1 om gegevens over beweging, status en verandering van een item in de toeleveringsketen (supply chain) vast te leggen en te delen \autocite{Devins}.
``Met behulp van deze waarden kunnen we real-life objecten omzetten in elektronisch opgeslagen informatie, waarna we dit kunnen communiceren met eindgebruikers.`` zegt \textcite{Devins}.
Door deze normen toe te passen kunnen we de traceerbaarheid van het product per proces garanderen inclusief de gewenste parameters die opgeslagen worden in ons grafiekmodel zoals tijd (wanneer) en temperatuur (hoe) waar nodig.
\subsubsection{Waarom EPCIS?}
Huidige Legacy Systemen maken gebruik van ERP, POS, WMS,\dots Vaak lopen deze niet real-time wat nadelig is als we direct iets willen weten over een product en hoge beschikbaarheid verwachten \autocite{Vieweger}.
Doordat iedereen eenzelfde norm gebruikt kunnen we dit schaalbaar houden en makkelijk uitbereiden.



% Hier beschrijf je de \emph{state-of-the-art} rondom je gekozen onderzoeksdomein, d.w.z.\ een inleidende, doorlopende tekst over het onderzoeksdomein van je bachelorproef. Je steunt daarbij heel sterk op de professionele \emph{vakliteratuur}, en niet zozeer op populariserende teksten voor een breed publiek. Wat is de huidige stand van zaken in dit domein, en wat zijn nog eventuele open vragen (die misschien de aanleiding waren tot je onderzoeksvraag!)?

% Je mag de titel van deze sectie ook aanpassen (literatuurstudie, stand van zaken, enz.). Zijn er al gelijkaardige onderzoeken gevoerd? Wat concluderen ze? Wat is het verschil met jouw onderzoek?

% Verwijs bij elke introductie van een term of bewering over het domein naar de vakliteratuur, bijvoorbeeld~\autocite{Hykes2013}! Denk zeker goed na welke werken je refereert en waarom.

% Draag zorg voor correcte literatuurverwijzingen! Een bronvermelding hoort thuis \emph{binnen} de zin waar je je op die bron baseert, dus niet er buiten! Maak meteen een verwijzing als je gebruik maakt van een bron. Doe dit dus \emph{niet} aan het einde van een lange paragraaf. Baseer nooit teveel aansluitende tekst op eenzelfde bron.

% Als je informatie over bronnen verzamelt in JabRef, zorg er dan voor dat alle nodige info aanwezig is om de bron terug te vinden (zoals uitvoerig besproken in de lessen Research Methods).

% Voor literatuurverwijzingen zijn er twee belangrijke commando's:
% \autocite{KEY} => (Auteur, jaartal) Gebruik dit als de naam van de auteur
%   geen onderdeel is van de zin.
% \textcite{KEY} => Auteur (jaartal)  Gebruik dit als de auteursnaam wel een
%   functie heeft in de zin (bv. ``Uit onderzoek door Doll & Hill (1954) bleek
%   ...'')

% Je mag deze sectie nog verder onderverdelen in subsecties als dit de structuur van de tekst kan verduidelijken.

%---------- Methodologie ------------------------------------------------------
\section{Methodologie}%
\label{sec:methodologie}
Eerst en vooral gaan we aan de slag met een json bestand dat data bevat van Arcelor Mittal waar events in opgeslagen staan.
Daarna zetten we dit om naar Cosmos DB en kunnen we via Grammlin API werken om een grafiek op te zetten met tijdsgebonden edges. Dit dient ervoor om te zorgen dat we in tijd terug kunnen om de historiek op te vragen en te kijken waar, wanneer en hoe een proces gebeurd is.
Daardoor kunnen we achteraf bij onder andere klachtenafhandeling terug kijken in het grafiekmodel wat er mogelijks anders was of fout is gelopen.

Dit grafiekmodel moet ook voldoen aan de normen volgens EPCIS en schaalbaar zijn, daarom gaan we zorgen dat de gebruikte data binnen het grafiekmodel zoals een node een id krijgt in de plaats van een statische naam.
Daardoor kan Tracked in latere fases of andere business cases dezelfde technieken gebruiken zonder veel aanpassingen te moeten doen op grote schaal.

Zodra we een duidelijk grafiekmodel hebben waarmee we makkelijk in het verleden kunnen kijken, kunnen we dit trainen met copilot van microsoft.
Daardoor kunnen we dan snel zoeken naar waar mogelijks anomalieën zijn in een proces door te vergelijken met andere batches die hetzelfde proces hebben doorlopen.
Als laatste kunnen we dan via copilot makkelijk een implementatie starten voor onder andere microsoft teams waarbij iemand van de klantenservice kan vragen wat er fout is gelopen en dat de chatbot zegt waar, wanneer en hoe er mogelijks een fout is geweest in het proces.
% - Eerste fase -> begrijpen wat tijdelijke transfereerbare graph techniek is. Werken met CosmosDB heeft grammling API embedded (apache gremlin (https://tinkerpop.apache.org/gremlin.html)). EPCIS Datamodel leren (events van AM transformeren naar Gremlin)
% - Tweede fase -> Aanleren copilot studio (microsoft) en hoe linken met graph en chatbot maken.

% Hier beschrijf je hoe je van plan bent het onderzoek te voeren. Welke onderzoekstechniek ga je toepassen om elk van je onderzoeksvragen te beantwoorden? Gebruik je hiervoor literatuurstudie, interviews met belanghebbenden (bv.~voor requirements-analyse), experimenten, simulaties, vergelijkende studie, risico-analyse, PoC, \ldots?

% Valt je onderwerp onder één van de typische soorten bachelorproeven die besproken zijn in de lessen Research Methods (bv.\ vergelijkende studie of risico-analyse)? Zorg er dan ook voor dat we duidelijk de verschillende stappen terug vinden die we verwachten in dit soort onderzoek!

% Vermijd onderzoekstechnieken die geen objectieve, meetbare resultaten kunnen opleveren. Enquêtes, bijvoorbeeld, zijn voor een bachelorproef informatica meestal \textbf{niet geschikt}. De antwoorden zijn eerder meningen dan feiten en in de praktijk blijkt het ook bijzonder moeilijk om voldoende respondenten te vinden. Studenten die een enquête willen voeren, hebben meestal ook geen goede definitie van de populatie, waardoor ook niet kan aangetoond worden dat eventuele resultaten representatief zijn.

% Uit dit onderdeel moet duidelijk naar voor komen dat je bachelorproef ook technisch voldoen\-de diepgang zal bevatten. Het zou niet kloppen als een bachelorproef informatica ook door bv.\ een student marketing zou kunnen uitgevoerd worden.

% Je beschrijft ook al welke tools (hardware, software, diensten, \ldots) je denkt hiervoor te gebruiken of te ontwikkelen.

% Probeer ook een tijdschatting te maken. Hoe lang zal je met elke fase van je onderzoek bezig zijn en wat zijn de concrete \emph{deliverables} in elke fase?

%---------- Verwachte resultaten ----------------------------------------------
\section{Verwacht resultaat, conclusie}%
\label{sec:verwachte_resultaten}
In dit onderzoek maken we startlaag aan een groter project waarbij het mogelijk is om uit te bereiden naar bijvoorbeeld een planningstool om de meest efficiënte route te zoeken.
Door deze chatbot verwachten we dat medewerkers van Arcelor Mittal eenvoudig een vraag kan stellen en een antwoord kan krijgen op een korte tijd.
Zonder deze optie moeten ze elk deel van het bedrijf handmatig opbellen en vragen wat er gebeurd is, dat zijn zaken die zeer tijdsslopend zijn en veel geld kosten.
Ik hoop ook dat we dit schaalbaar kunnen opstellen en performant kunnen houden waardoor het ook kan gebruikt worden op landelijk of internationaal niveau.
Door de implementatie van dit systeem hoop ik dat we klanttevredenheid en besparingen binnen een bedrijf (in ons geval Arcelor Mittal Gent) optimaliseren
% - Kan een medewerken van Arcelor Mittal op een eenvoudige manier een antwoord vinden op zijn vraag en wat is de tijdswints & geldbesparing dat ermee gepaard gaat.
% - Kunnen we op een bepaalde schaalgrootte de grafiek maken en performant houden
% BusinessUC -> Moet opgenomen worden in verwahte resultaat kost/besparing samenleggen.


% Hier beschrijf je welke resultaten je verwacht. Als je metingen en simulaties uitvoert, kan je hier al mock-ups maken van de grafieken samen met de verwachte conclusies. Benoem zeker al je assen en de onderdelen van de grafiek die je gaat gebruiken. Dit zorgt ervoor dat je concreet weet welk soort data je moet verzamelen en hoe je die moet meten.

% Wat heeft de doelgroep van je onderzoek aan het resultaat? Op welke manier zorgt jouw bachelorproef voor een meerwaarde?

% Hier beschrijf je wat je verwacht uit je onderzoek, met de motivatie waarom. Het is \textbf{niet} erg indien uit je onderzoek andere resultaten en conclusies vloeien dan dat je hier beschrijft: het is dan juist interessant om te onderzoeken waarom jouw hypothesen niet overeenkomen met de resultaten.



%%---------- Andere bijlagen --------------------------------------------------
% TODO: Voeg hier eventuele andere bijlagen toe. Bv. als je deze BP voor de
% tweede keer indient, een overzicht van de verbeteringen t.o.v. het origineel.
%\input{...}

%%---------- Backmatter, referentielijst ---------------------------------------

\backmatter{}

\setlength\bibitemsep{2pt} %% Add Some space between the bibliograpy entries
\printbibliography[heading=bibintoc]

\end{document}
