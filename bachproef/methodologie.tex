%%=============================================================================
%% Methodologie
%%=============================================================================

\chapter{\IfLanguageName{dutch}{Methodologie}{Methodology}}%
\label{ch:methodologie}
\section{Dataverwerking}
\subsection{Technologieën voor databases}
In het eerste deel van deze bachelorproef wordt er een literatuurstudie uitgevoerd over de verschillende technologieën die we gebruiken om onze data op te slaan en te verwerken.
Hier hebben we de werking en efficiëntie van CosmosDB onderzocht. Daarnaast hebben we ook gekeken naar de werking van Gremlin API en hoe we deze kunnen gebruiken om een grafiekmodel op te zetten.
De data die we ontvangen is vrij ingewikkelde data. Dit is namelijk een boomstructuur uit SAP die we via een JSON bestand hebben ontvangen. 
Door gebruik te maken van Python en Javascript gaan we deze data omzetten en gebruiken voor onze chatbot.

\subsection{SAP-data verwerken}
De SAP data die we ontvangen is een boomstructuur die we ontvangen hebben in een JSON bestand. Deze data gaan we aan de hand van een JavaScript programma en NodeJS inlezen in CosmosDB omzetten naar een JSONLD waar de data gelinkt wordt aan elkaar.
Deze JSONLD kunnen we daarna als grafiekmodel laten opbouwen in CosmosDB door gebruik te maken van de Gremlin API.\@
De vertaling van de datacodes uit SAP zijn vrij lastig en complex aangezien elke sleutel-waarde paar een specifieke code of waarde is. Daarom is het belangrijk dat we deze data goed analyseren en vertalen naar een leesbare vorm voor het grafiekmodel.
Hiervoor hebben we binnen ArcelorMittal verschillende afdelingen gecontacteerd om de datacodes te kunnen vertalen. Door middel van een excel bestand waarin deze codes en waarden staan in verschillende talen, kunnen we deze sleutelwaarden vertalen naar begrijpbare eigenschappen.
Deze waarden komen als eigenschappen in de knopen van ons grafiekmodel terecht, waarna we deze kunnen opvragen met onze chatbot.

\section{Proof of concept}
\subsection{Opzetten van een grafiekmodel}
In het tweede deel van deze bachelorproef wordt er een proof of concept opgesteld van het grafiekmodel waarbij we de SAP data van ArcelorMittal Gent zullen gebruiken.
Zoals eerder vermeld hebben we de van JSON data omgezet naar GS1-standaarden in een JSONLD-bestand. Dit bestand dat opgemaakt is met schema.org, GS1 en EPCIS-events wordt ingeladen in CosmosDB via NodeJS en Gremlin API.\@
Daardoor ontstaat een grafiekmodel waarin we de data kunnen visualiseren en analyseren door de relaties tussen de verschillende producten en processen.
Dit grafiekmodel kan gevisualiseerd worden met behulp van Graph Explorer of andere visualisatie tools.

\subsection{Opzetten van een chatbot}
In het derde deel van deze bachelorproef wordt er een chatbot opgesteld die de data van het grafiekmodel kan bevragen en analyseren.
De input van dit model is een vraag die de gebruiker stelt, deze vraag wordt vertaald naar een Gremlin query die de data in CosmosDB zal bevragen.
Die kan dan op zijn beurt de relevante knopen en relaties vinden, die worden verwerkt en geformatteerd in leesbare tekst.
De chatbot is een iteratief proces, waarbij hij in de eerste iteratie de query maakt en in de tweede iteratie de resultaten teruggeeft aan de gebruiker.
Tijdens dit iteratief proces maken we gebruik van een Retrieval Augmented Generation (RAG) model dat de resultaten van de Gremlin query kan verwerken en formatteren naar een leesbare tekst.
Er zijn twee mogelijkheden om een query op te halen. De eerste methode maakt gebruik van elasticsearch die in een docker container beschikbaar is. 
Daarbij gebruiken we een CSV bestand met voorbeeld vragen, die in een database geïndexeerd worden, om sneller en meer accuraat de juiste query te vinden.
De tweede methode maakt gebruik van de LLM indien er geen bijhorende query gevonden kan worden in de elasticsearch.
Daarnaast valideren we de query die we terugkrijgen van het Large Language Model, zodat de query uitvoerbaar is in CosmosDB.\@

\subsection{Structuur chatbot}
Voor onze chatbot maken we gebruik van codellama voor het genereren van een Gremlin query en een Phi4 model voor het samenvatten van de data uit de database.
Deze modellen worden gebruikt via Ollama zodat we een lokale versie gebruiken die niet verbonden is met het internet. Dit omdat we zeker willen zijn dat ons model geen data bijhoudt via het internet voor bijvoorbeeld het trainen van zijn eigen model.
Als voorbeeld van zo'n model die zichzelf trained hebben we OpenAI die een model heeft dat getraind is op een grote dataset, maar deze data wordt ook gebruikt om het model te verbeteren. 
Dit is niet de bedoeling voor ons model, want we willen zeker geen datalek veroorzaken.
We gaan met de chatbot een aantal vragen beantwoorden die gerelateerd zijn aan het productieproces van ArcelorMittal Gent. Dit zijn vragen zoals: ``Welke fabrieken staan er in Gent'' of ``Geef alle kranen met een melding op de motor''.
Dit probleem lossen we op door het iteratief proces binnen onze chatbot waarbij we de vraag doorsturen naar de chatbot om het te beantwoorden.
%% TODO: In dit hoofstuk geef je een korte toelichting over hoe je te werk bent
%% gegaan. Verdeel je onderzoek in grote fasen, en licht in elke fase toe wat
%% de doelstelling was, welke deliverables daar uit gekomen zijn, en welke
%% onderzoeksmethoden je daarbij toegepast hebt. Verantwoord waarom je
%% op deze manier te werk gegaan bent.
%% 
%% Voorbeelden van zulke fasen zijn: literatuurstudie, opstellen van een
%% requirements-analyse, opstellen long-list (bij vergelijkende studie),
%% selectie van geschikte tools (bij vergelijkende studie, "short-list"),
%% opzetten testopstelling/PoC, uitvoeren testen en verzamelen
%% van resultaten, analyse van resultaten, ...
%%
%% !!!!! LET OP !!!!!
%%
%% Het is uitdrukkelijk NIET de bedoeling dat je het grootste deel van de corpus
%% van je bachelorproef in dit hoofstuk verwerkt! Dit hoofdstuk is eerder een
%% kort overzicht van je plan van aanpak.
%%
%% Maak voor elke fase (behalve het literatuuronderzoek) een NIEUW HOOFDSTUK aan
%% en geef het een gepaste titel.



