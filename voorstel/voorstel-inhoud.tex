%---------- Inleiding ---------------------------------------------------------

% TODO: Is dit voorstel gebaseerd op een paper van Research Methods die je
% vorig jaar hebt ingediend? Heb je daarbij eventueel samengewerkt met een
% andere student?
% Zo ja, haal dan de tekst hieronder uit commentaar en pas aan.

%\paragraph{Opmerking}

% Dit voorstel is gebaseerd op het onderzoeksvoorstel dat werd geschreven in het
% kader van het vak Research Methods dat ik (vorig/dit) academiejaar heb
% uitgewerkt (met medesturent VOORNAAM NAAM als mede-auteur).
% 

\section{Inleiding}%
\label{sec:inleiding}
Binnen de Arcelor Mittal Gent bestaan er veel verschillende processen die verspreid zijn over meerdere afdelingen.
Hierbij bevat elke afdeling een deel van het proces om van grondstof tot een afgewerkt product te komen. 
Tussen elke afdeling bevindt zich een deel van de toeleveringsketen, waarbij het resultaat van de ene afdeling het beginpunt is voor de volgende afdeling.
Dit zorgt ervoor dat iedere stap zijn invloed heeft op de kwaliteit van het eindresultaat.

Elke afdeling is opgebouwd rond een specifiek proces, zoals de hoogoven die ruwe grondstoffen omzet in ruw ijzer en de staalfabriek die ruw ijzer in staal verandert.
Omdat deze afdelingen zeer specifieke processen hebben, hebben ze ook zeer specifieke behoeften en datamodellen.
Dit maakt analyse of onderzoek over afdelingen heen zeer complex en tijdsrovend.

De scope van deze bachelorproef is tweedelig, waarbij deel één eruit bestaat om samen met Tracked vanuit een gestandariseerd datamodel die zichtbaarheid events bijhoudt om te zetten naar een grafiekmodel.
Deel twee bouwt verder voort op dit grafiekmodel om dan een LLM (large language model) te trainen die eenvoudig informatie eruit kan halen en teruggeven.

De centrale vraag die hier gesteld wordt is: ``Hoe we efficiënt en snel grafiekmodellering kunnen toepassen om een LLM te ontwikkelen die in staat is om het waar, wanneer, wat en hoe van gebeurtenissen binnen een proces vast te stellen, ter ondersteuning van klachtenafhandeling bij productfouten.``
%---------- Stand van zaken ---------------------------------------------------

\section{Literatuurstudie}%
\label{sec:literatuurstudie}

\subsection{Cosmos DB}%
\subsubsection{Waarom Cosmos DB gebruiken}
Cosmos DB is een NoSQL-database van microsoft, deze heeft een lage latentie, multi-query-api die makkelijk grote hoeveelheden data kan verwerken en heeft een grote beschikbaarheid zegt~\textcite{Put2020} wat zeer belangrijk is in ons project.
Naast deze feiten is CosmosDB horizontaal schaalbaar wat betekent dat we op hoogtepunten een miljoen lees- en schrijfaanvragen kunnen verwerken door het nodige aantal servers toe te voegen.
De hoge beschikbaarheid wordt gegarandeerd door replicatie waardoor we snel kunnen overschakelen als er iets fout gaat in de database.

\subsection{Grafiek modellering}
Cosmos DB biedt ook grafiek modellering aan met behulp van Gremlin. Gremlin is een grafiek database service waarbij we grote grafieken kunnen opslaan die bestaan uit vertices en edges \autocite{Microsoft2024}.
Dit gebeurt met een snelheid van miliseconden waardoor we een snelle verwerkingstijd zullen kunnen neerzetten.
Ook is Gremlin schaalbaar en kan de consistentie level gekozen worden om een balans te vinden tussen consistentie, beschikbaarheid en latentie.
Binnen zo een grafiek model hebben we Vertices (of nodes) die een persoon, plaats of event beschrijven zoals in ons geval bijvoorbeeld een slab die verplaatst wordt van A naar B.

\subsubsection{Verschil transformatie en aggregatie}
Binnen ons project kunnen bestaansvormen van de producten veranderen. Zo hebben we transformatie waarbij de cokes bijvoorbeeld omgezet worden in vloeibaar staal, dit is niet omkeerbaar.
Bij aggregatie kunnen we terug naar de oorspronkelijke bestaandsvorm zoals bijvoorbeeld van warmwalserij naar koudwalserij waarbij de platen dunner worden en eventueel een roestvrije laag krijgen \autocite{ArcelorMittal2024}.
In onze grafiek modellering moeten we gebruik maken van tijdelijke edges om ook terug te kunnen kijken naar wat voor het transformatie- of aggregatieproces gebeurd is om anomalieën terug te vinden \autocite{JaewookByun2020}.


\subsection{EPCIS}
Voor dit onderzoek moet alles voldoen aan de EPCIS (Electronic Product Code Information Services) waarden, dit zijn de wat, wanneer, waar, waarom en hoe. 
Deze waarden zijn ontwikkeld door GS1 om gegevens over beweging, status en verandering van een item in de toeleveringsketen (supply chain) vast te leggen en te delen \autocite{Devins}.
``Met behulp van deze waarden kunnen we real-life objecten omzetten in elektronisch opgeslagen informatie, waarna we dit kunnen communiceren met eindgebruikers.`` zegt \textcite{Devins}.
Door deze normen toe te passen kunnen we de traceerbaarheid van het product per proces garanderen inclusief de gewenste parameters die opgeslagen worden in ons grafiekmodel zoals tijd (wanneer) en temperatuur (hoe) waar nodig.
\subsubsection{Waarom EPCIS?}
Huidige Legacy Systemen maken gebruik van ERP, POS, WMS,\dots Vaak lopen deze niet real-time wat nadelig is als we direct iets willen weten over een product en hoge beschikbaarheid verwachten \autocite{Vieweger}.
Doordat iedereen eenzelfde norm gebruikt kunnen we dit schaalbaar houden en makkelijk uitbereiden.



% Hier beschrijf je de \emph{state-of-the-art} rondom je gekozen onderzoeksdomein, d.w.z.\ een inleidende, doorlopende tekst over het onderzoeksdomein van je bachelorproef. Je steunt daarbij heel sterk op de professionele \emph{vakliteratuur}, en niet zozeer op populariserende teksten voor een breed publiek. Wat is de huidige stand van zaken in dit domein, en wat zijn nog eventuele open vragen (die misschien de aanleiding waren tot je onderzoeksvraag!)?

% Je mag de titel van deze sectie ook aanpassen (literatuurstudie, stand van zaken, enz.). Zijn er al gelijkaardige onderzoeken gevoerd? Wat concluderen ze? Wat is het verschil met jouw onderzoek?

% Verwijs bij elke introductie van een term of bewering over het domein naar de vakliteratuur, bijvoorbeeld~\autocite{Hykes2013}! Denk zeker goed na welke werken je refereert en waarom.

% Draag zorg voor correcte literatuurverwijzingen! Een bronvermelding hoort thuis \emph{binnen} de zin waar je je op die bron baseert, dus niet er buiten! Maak meteen een verwijzing als je gebruik maakt van een bron. Doe dit dus \emph{niet} aan het einde van een lange paragraaf. Baseer nooit teveel aansluitende tekst op eenzelfde bron.

% Als je informatie over bronnen verzamelt in JabRef, zorg er dan voor dat alle nodige info aanwezig is om de bron terug te vinden (zoals uitvoerig besproken in de lessen Research Methods).

% Voor literatuurverwijzingen zijn er twee belangrijke commando's:
% \autocite{KEY} => (Auteur, jaartal) Gebruik dit als de naam van de auteur
%   geen onderdeel is van de zin.
% \textcite{KEY} => Auteur (jaartal)  Gebruik dit als de auteursnaam wel een
%   functie heeft in de zin (bv. ``Uit onderzoek door Doll & Hill (1954) bleek
%   ...'')

% Je mag deze sectie nog verder onderverdelen in subsecties als dit de structuur van de tekst kan verduidelijken.

%---------- Methodologie ------------------------------------------------------
\section{Methodologie}%
\label{sec:methodologie}
Eerst en vooral gaan we aan de slag met een json bestand dat data bevat van Arcelor Mittal waar events in opgeslagen staan.
Daarna zetten we dit om naar Cosmos DB en kunnen we via Grammlin API werken om een grafiek op te zetten met tijdsgebonden edges. Dit dient ervoor om te zorgen dat we in tijd terug kunnen om de historiek op te vragen en te kijken waar, wanneer en hoe een proces gebeurd is.
Daardoor kunnen we achteraf bij onder andere klachtenafhandeling terug kijken in het grafiekmodel wat er mogelijks anders was of fout is gelopen.

Dit grafiekmodel moet ook voldoen aan de normen volgens EPCIS en schaalbaar zijn, daarom gaan we zorgen dat de gebruikte data binnen het grafiekmodel zoals een node een id krijgt in de plaats van een statische naam.
Daardoor kan Tracked in latere fases of andere business cases dezelfde technieken gebruiken zonder veel aanpassingen te moeten doen op grote schaal.

Zodra we een duidelijk grafiekmodel hebben waarmee we makkelijk in het verleden kunnen kijken, kunnen we dit trainen met copilot van microsoft.
Daardoor kunnen we dan snel zoeken naar waar mogelijks anomalieën zijn in een proces door te vergelijken met andere batches die hetzelfde proces hebben doorlopen.
Als laatste kunnen we dan via copilot makkelijk een implementatie starten voor onder andere microsoft teams waarbij iemand van de klantenservice kan vragen wat er fout is gelopen en dat de chatbot zegt waar, wanneer en hoe er mogelijks een fout is geweest in het proces.
% - Eerste fase -> begrijpen wat tijdelijke transfereerbare graph techniek is. Werken met CosmosDB heeft grammling API embedded (apache gremlin (https://tinkerpop.apache.org/gremlin.html)). EPCIS Datamodel leren (events van AM transformeren naar Gremlin)
% - Tweede fase -> Aanleren copilot studio (microsoft) en hoe linken met graph en chatbot maken.

% Hier beschrijf je hoe je van plan bent het onderzoek te voeren. Welke onderzoekstechniek ga je toepassen om elk van je onderzoeksvragen te beantwoorden? Gebruik je hiervoor literatuurstudie, interviews met belanghebbenden (bv.~voor requirements-analyse), experimenten, simulaties, vergelijkende studie, risico-analyse, PoC, \ldots?

% Valt je onderwerp onder één van de typische soorten bachelorproeven die besproken zijn in de lessen Research Methods (bv.\ vergelijkende studie of risico-analyse)? Zorg er dan ook voor dat we duidelijk de verschillende stappen terug vinden die we verwachten in dit soort onderzoek!

% Vermijd onderzoekstechnieken die geen objectieve, meetbare resultaten kunnen opleveren. Enquêtes, bijvoorbeeld, zijn voor een bachelorproef informatica meestal \textbf{niet geschikt}. De antwoorden zijn eerder meningen dan feiten en in de praktijk blijkt het ook bijzonder moeilijk om voldoende respondenten te vinden. Studenten die een enquête willen voeren, hebben meestal ook geen goede definitie van de populatie, waardoor ook niet kan aangetoond worden dat eventuele resultaten representatief zijn.

% Uit dit onderdeel moet duidelijk naar voor komen dat je bachelorproef ook technisch voldoen\-de diepgang zal bevatten. Het zou niet kloppen als een bachelorproef informatica ook door bv.\ een student marketing zou kunnen uitgevoerd worden.

% Je beschrijft ook al welke tools (hardware, software, diensten, \ldots) je denkt hiervoor te gebruiken of te ontwikkelen.

% Probeer ook een tijdschatting te maken. Hoe lang zal je met elke fase van je onderzoek bezig zijn en wat zijn de concrete \emph{deliverables} in elke fase?

%---------- Verwachte resultaten ----------------------------------------------
\section{Verwacht resultaat, conclusie}%
\label{sec:verwachte_resultaten}
In dit onderzoek maken we startlaag aan een groter project waarbij het mogelijk is om uit te bereiden naar bijvoorbeeld een planningstool om de meest efficiënte route te zoeken.
Door deze chatbot verwachten we dat medewerkers van Arcelor Mittal eenvoudig een vraag kan stellen en een antwoord kan krijgen op een korte tijd.
Zonder deze optie moeten ze elk deel van het bedrijf handmatig opbellen en vragen wat er gebeurd is, dat zijn zaken die zeer tijdsslopend zijn en veel geld kosten.
Ik hoop ook dat we dit schaalbaar kunnen opstellen en performant kunnen houden waardoor het ook kan gebruikt worden op landelijk of internationaal niveau.
Door de implementatie van dit systeem hoop ik dat we klanttevredenheid en besparingen binnen een bedrijf (in ons geval Arcelor Mittal Gent) optimaliseren
% - Kan een medewerken van Arcelor Mittal op een eenvoudige manier een antwoord vinden op zijn vraag en wat is de tijdswints & geldbesparing dat ermee gepaard gaat.
% - Kunnen we op een bepaalde schaalgrootte de grafiek maken en performant houden
% BusinessUC -> Moet opgenomen worden in verwahte resultaat kost/besparing samenleggen.


% Hier beschrijf je welke resultaten je verwacht. Als je metingen en simulaties uitvoert, kan je hier al mock-ups maken van de grafieken samen met de verwachte conclusies. Benoem zeker al je assen en de onderdelen van de grafiek die je gaat gebruiken. Dit zorgt ervoor dat je concreet weet welk soort data je moet verzamelen en hoe je die moet meten.

% Wat heeft de doelgroep van je onderzoek aan het resultaat? Op welke manier zorgt jouw bachelorproef voor een meerwaarde?

% Hier beschrijf je wat je verwacht uit je onderzoek, met de motivatie waarom. Het is \textbf{niet} erg indien uit je onderzoek andere resultaten en conclusies vloeien dan dat je hier beschrijft: het is dan juist interessant om te onderzoeken waarom jouw hypothesen niet overeenkomen met de resultaten.

